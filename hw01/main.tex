\documentclass[11pt]{article}
\usepackage{amsfonts}
\usepackage{amsmath}

\newcommand{\N}{\mathbb{N}}
\newcommand{\Z}{\mathbb{Z}}
\newcommand{\Q}{\mathbb{Q}}
\newcommand{\R}{\mathbb{R}}
\renewcommand{\P}{{\cal P}}

\pagestyle{empty}


\begin{document}

\setlength{\parindent}{0pt}
\setlength{\parskip}{9pt}


\section*{Math 351: Homework 1 (Due September 14)}
\section*{Name: Jack Ellert-Beck}

\bigskip

\subsection*{Section 0.2}
\subsection*{Problem 1}

Prove the following equivalences:

a) $\lnot (P \lor Q) \equiv (\neg P \land \lnot Q)$

We can write a truth table that lists the truth values of
$\lnot (P \lor Q)$ and $(\neg P \land \lnot Q)$ for all truth values of $P$ and $Q$:

$$
\begin{array}{|c|c|c|c|c|c|c|}
P & Q & \lnot P & \lnot Q & P \lor Q & \lnot (P \lor Q) & (\lnot P \land \lnot Q) \\
\hline
T & T & F & F & T & \bf F & \bf F \\
T & F & F & T & T & \bf F & \bf F \\
F & T & T & F & T & \bf F & \bf F \\
F & F & T & T & F & \bf T & \bf T \\
\end{array}
$$

Since the two statements have the same truth value in all cases, they are equivalent.

b) $\lnot (P \land Q) \equiv (\lnot P \lor \lnot Q)$

We can write a truth value to test every case:

$$
\begin{array}{|c|c|c|c|c|c|c|}
P & Q & \lnot P & \lnot Q & P \land Q & \lnot (P \land Q) & (\lnot P \lor \lnot Q) \\
\hline
T & T & F & F & T & \bf F & \bf F \\
T & F & F & T & F & \bf T & \bf T \\
F & T & T & F & F & \bf T & \bf T \\
F & F & T & T & F & \bf T & \bf T \\
\end{array}
$$

Since both statements have the same truth value for any value of $P$ or $Q$, they
are equivalent.

\subsection*{Problem 2}

Prove that $P \implies Q \equiv (\lnot P) \lor Q$. Deduce that the negation of
$P \implies Q$ is $P \land (\lnot Q)$.

We write a truth table to show that $P \implies Q$ has the same truth value as
$(\lnot P)\lor Q$ in all cases:

$$
\begin{array}{|c|c|c|c|c|}
P & Q & \lnot P & (\lnot P) \lor Q & P \implies Q \\
\hline
T & T & F & \bf T & \bf T \\
T & F & F & \bf F & \bf F \\
F & T & T & \bf T & \bf T \\
F & F & T & \bf T & \bf T \\
\end{array}
$$

Since $P \implies Q$ is equivalent to $(\lnot P) \lor Q$, their negations will be
equivalent. So we negate  $(\lnot P) \lor Q$ and manipulate using equivalences from
Problem 1 to find the negation:

$$
\begin{array}{rl}
\lnot (P \implies Q) &= \lnot (\lnot P \lor Q) \\
&= \lnot(\lnot P \lor \lnot (\lnot Q)) \\
&= \lnot (\lnot (P \land (\lnot Q))) \\
&= P \land (\lnot Q).
\end{array}
$$

We also used the fact that $\lnot (\lnot P) \equiv P$, which we now show with a
truth table:

$$
\begin{array}{|c|c|c|}
P & \lnot P & \lnot (\lnot P) \\
\hline
\bf T & F & \bf T \\
\bf F & T & \bf F \\
\end{array}
$$

\subsection*{Problem 3}

Find the negation of the following statements.

a) $\lnot(P \land \lnot Q) \lor R$

$$
\begin{array}{rl}
\lnot (\lnot(P \land \lnot Q) \lor R) &= (P \land \lnot Q) \land \lnot R \\
&= P \land \lnot Q \land \lnot R.
\end{array}
$$

b) $P \implies (Q \lor R)$

$$
\begin{array}{rl}
\lnot (P \implies (Q \lor R)) &= P \land \lnot (Q \lor R) \\
&= P \land \lnot Q \land \lnot R.
\end{array}
$$

c) $\lnot (P \lor Q) \implies (R \lor S)$

$$
\begin{array}{rl}
\lnot (\lnot (P \lor Q) \implies (R \lor S))
&= \lnot ((P \lor Q) \lor (R \lor S)) \\
&= \lnot (P \lor Q) \land \lnot (R \lor S) \\
&= \lnot P \land \lnot Q \land \lnot R \land \lnot S.
\end{array}
$$
\subsection*{Problem 5}

Prove that an implication is equivalent to its contrapositive.

The contrapositive of $P \implies Q$ is the statement
$\lnot Q \implies \lnot P$. We can manipulate the statement $\lnot P \lor Q$, noting
that it is equivalent to $P \implies Q$:
$$
\begin{array}{rl}
\lnot P \lor Q &= Q \lor \lnot P \\
&= \lnot (\lnot Q) \lor \lnot P \\
&= \lnot Q \implies \lnot P.
\end{array}
$$

\subsection*{Problem 6}
An example of an implication whose converse is not true:

If polygon $ABCD$ is a square, then its interior angles are all 90 degrees.

An example of an implcation whose converse is true:
\[ (x = 4) \implies (x+2 = 6) \]
\subsection*{Problem 7}

\subsection*{Section 0.3}
\subsection*{Problem 1}
\subsection*{Problem 2}
\subsection*{Problem 3}
\subsection*{Problem 4}
\subsection*{Problem 5}
\subsection*{Problem 17}


\end{document}
