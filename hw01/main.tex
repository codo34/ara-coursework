\documentclass[11pt]{article}
\usepackage{amsfonts}
\usepackage{amsmath}
\usepackage{amssymb}

\newcommand{\N}{\mathbb{N}}
\newcommand{\Z}{\mathbb{Z}}
\newcommand{\Q}{\mathbb{Q}}
\newcommand{\R}{\mathbb{R}}
\renewcommand{\P}{{\cal P}}

\pagestyle{empty}


\begin{document}

\setlength{\parindent}{0pt}
\setlength{\parskip}{9pt}


\section*{Math 351: Homework 1 (Due September 14)}
\section*{Name: Jack Ellert-Beck}

\bigskip

\subsection*{Section 0.2}
\subsection*{Problem 1}

Prove the following equivalences:

a) $\lnot (P \lor Q) \equiv (\neg P \land \lnot Q)$

We can write a truth table that lists the truth values of
$\lnot (P \lor Q)$ and $(\neg P \land \lnot Q)$ for all truth values of $P$ and $Q$:

$$
\begin{array}{|c|c|c|c|c|c|c|}
P & Q & \lnot P & \lnot Q & P \lor Q & \lnot (P \lor Q) & (\lnot P \land \lnot Q) \\
\hline
T & T & F & F & T & \bf F & \bf F \\
T & F & F & T & T & \bf F & \bf F \\
F & T & T & F & T & \bf F & \bf F \\
F & F & T & T & F & \bf T & \bf T \\
\end{array}
$$

Since the two statements have the same truth value in all cases, they are equivalent.

b) $\lnot (P \land Q) \equiv (\lnot P \lor \lnot Q)$

We can write a truth value to test every case:

$$
\begin{array}{|c|c|c|c|c|c|c|}
P & Q & \lnot P & \lnot Q & P \land Q & \lnot (P \land Q) & (\lnot P \lor \lnot Q) \\
\hline
T & T & F & F & T & \bf F & \bf F \\
T & F & F & T & F & \bf T & \bf T \\
F & T & T & F & F & \bf T & \bf T \\
F & F & T & T & F & \bf T & \bf T \\
\end{array}
$$

Since both statements have the same truth value for any value of $P$ or $Q$, they
are equivalent.

\subsection*{Problem 2}

Prove that $P \implies Q \equiv (\lnot P) \lor Q$. Deduce that the negation of
$P \implies Q$ is $P \land (\lnot Q)$.

We write a truth table to show that $P \implies Q$ has the same truth value as
$(\lnot P)\lor Q$ in all cases:

$$
\begin{array}{|c|c|c|c|c|}
P & Q & \lnot P & (\lnot P) \lor Q & P \implies Q \\
\hline
T & T & F & \bf T & \bf T \\
T & F & F & \bf F & \bf F \\
F & T & T & \bf T & \bf T \\
F & F & T & \bf T & \bf T \\
\end{array}
$$

Since $P \implies Q$ is equivalent to $(\lnot P) \lor Q$, their negations will be
equivalent. So we negate  $(\lnot P) \lor Q$ and manipulate using equivalences from
Problem 1 to find the negation:

$$
\begin{array}{rl}
\lnot (P \implies Q) &= \lnot (\lnot P \lor Q) \\
&= \lnot(\lnot P \lor \lnot (\lnot Q)) \\
&= \lnot (\lnot (P \land (\lnot Q))) \\
&= P \land (\lnot Q).
\end{array}
$$

We also used the fact that $\lnot (\lnot P) \equiv P$, which we now show with a
truth table:

$$
\begin{array}{|c|c|c|}
P & \lnot P & \lnot (\lnot P) \\
\hline
\bf T & F & \bf T \\
\bf F & T & \bf F \\
\end{array}
$$

\subsection*{Problem 3}

Find the negation of the following statements.

a) $\lnot(P \land \lnot Q) \lor R$

$$
\begin{array}{rl}
\lnot (\lnot(P \land \lnot Q) \lor R) &= (P \land \lnot Q) \land \lnot R \\
&= P \land \lnot Q \land \lnot R.
\end{array}
$$

b) $P \implies (Q \lor R)$

$$
\begin{array}{rl}
\lnot (P \implies (Q \lor R)) &= P \land \lnot (Q \lor R) \\
&= P \land \lnot Q \land \lnot R.
\end{array}
$$

c) $\lnot (P \lor Q) \implies (R \lor S)$

$$
\begin{array}{rl}
\lnot (\lnot (P \lor Q) \implies (R \lor S))
&= \lnot ((P \lor Q) \lor (R \lor S)) \\
&= \lnot (P \lor Q) \land \lnot (R \lor S) \\
&= \lnot P \land \lnot Q \land \lnot R \land \lnot S.
\end{array}
$$
\subsection*{Problem 5}

Prove that an implication is equivalent to its contrapositive.

The contrapositive of $P \implies Q$ is the statement
$\lnot Q \implies \lnot P$. We can manipulate the statement $\lnot P \lor Q$, noting
that it is equivalent to $P \implies Q$:
$$
\begin{array}{rl}
\lnot P \lor Q &= Q \lor \lnot P \\
&= \lnot (\lnot Q) \lor \lnot P \\
&= \lnot Q \implies \lnot P.
\end{array}
$$

\subsection*{Problem 6}
An example of an implication whose converse is not true:

If polygon $ABCD$ is a square, then its interior angles are all 90 degrees.

An example of an implcation whose converse is true:
\[ (x = 4) \implies (x+2 = 6) \]
\subsection*{Problem 7}
Negate the following statements:

a) For every ice cream flavor there is a pie that goes with that flavor

There exists an ice cream flavor for which there are no pies that go with it.

b) For every race car there is a driver who can drive that car.

There exists at least one race car with no driver who can drive it.

c) There exists a race car that every driver can drive.

For each race car there exists at least one driver who cannot drive it.

d) There exists a driver that can drive every race car.

For all race car drivers there exists at least one car they cannot drive.

\subsection*{Section 0.3}
\subsection*{Problem 1}

Let $A, B \subseteq X$. Which of the statements is equivalent to 
$A \cup B \neq \varnothing $?

Answer: (b) $A \neq \varnothing \lor B \neq \varnothing $.
For the union of $A$ and $B$ to be nonempty it is sufficient for at least one of $A$ or $B$
to be nonempty.

\subsection*{Problem 2}

Let $A, B \subseteq X$. Which of the statements is equivalent to
$A \cap B = \varnothing $?

Answer: (b) $A = \varnothing \lor B = \varnothing$.
If any one of $A$ or $B$ is empty, then their intersection will be empty.

\subsection*{Problem 3}
Let $A$ and $B$ be sets. Prove that $A = (A \cap B) \cup (A \setminus B)$.

We consider the definitions of $A \cap B$ and $A \setminus B$. The set $A \cap B$
is the set of all elements of $A$ that are also in $B$. The set $A \setminus B$ is the
set of all elements of $A$ that are not in $B$. The union of these two sets,
$(A \cap B) \cup (A \setminus B)$, is the set of all elements of $A$ that are also
in $B$ or that are not also in $B$. Since everything is either in $B$ or not in $B$,
this set contains all elements of $A$. Thus, $A = (A \cap B) \cup (A \setminus B)$. 

\subsection*{Problem 4}

Let $A$ and $B$ be sets. Prove that $A \setminus (A \setminus B) = B$.

The premise is actually false. Let $A = \{1, 2\}$ and $B = \{2, 3\}$. We can see that
$A \setminus B = \{1\}$ and that $A \setminus \{1\} = \{2\} \neq \{2, 3\}$.

If instead the condition had been added that $B \subseteq A$, it would have been a good
problem.

\subsection*{Problem 5}

Let $A, B \subseteq X$. Prove that $A \subseteq B \iff B^c \subseteq A^c$.

We first prove that $A \subseteq B \implies B^c \subseteq A^c$. $A \subseteq B$ means that
all elements of $A$ are also in $B$. Also, by definition, $x \in B^c \implies x \notin B$. 
So, $\forall x \notin B^c$, $x$ cannot be in $A$, which means that $x \in A^c$. Thus, 
$B^c \subseteq A^c$. 

We now prove the other direction, that $B^c \subseteq A^c \implies A \subseteq B$.
Note that any $x \in B^c$ is also in $A^c$. In other words, $x$ cannot be in $B^c$ and
also be in $A$. So, any element of $A$ is also in $B$, which means $A \subseteq B$.

\subsection*{Problem 17}

Let $A$ and $B$ be two sets such that $A \subseteq B$. Prove that $\P (A) \subseteq \P (B)$.

Let $S$ be any element of $\P (A)$. By the definition of $\P (A)$, $S \subseteq A$.
Since $A \subseteq B$, every element of $S$ is in $B$ as well, so $S \subseteq B$.
Hence, $S \in \P (B)$. So, every $S \in \P (A)$ is also in $\P (B)$, which means
$\P (A) \subseteq \P (B)$. 

\subsection*{Section 0.4}
\subsection*{Problem 10}

Let $f: X \to Y$ be a function.

a) Prove that $f(f^{-1}(B)) \subseteq B$ for all $B \subseteq Y$.

Take any $y \in f(f^{-1}(B))$. This means that $y=f(x)$ for some $x \in f^{-1}(B)$.
By the definition of $f^{-1}(B)$, $f(x) \in B$, so $y \in B$. Thus,
$f(f^{-1}(B)) \subseteq B$. 

b) Prove that $f(f^{-1}(B)) = B$ when $f$ is surjective.

We have already shown that $f(f^{-1}(B)) \subseteq B$ in general, so it is also true
if $f$ is surjective. We will show that if $f$ is surjective, then 
$B \subseteq f(f^{-1}(B))$, and thus that $f(f^{-1}(B)) = B$. For $f$ to be surjective,
it is necessary that for all $y \in Y$ there exists some $x \in X$ such that $f(x) = y$.
In particular, for all $y \in B$ there exists an $x \in f^{-1}(B)$ where $f(x) = y$.
Hence, $y \in f(f^{-1}(B))$ for any $y \in B$. So, when $f$ is surjective,
$B \subseteq f(f^{-1}(B))$, and thus $B = f(f^{-1}(B))$.

\subsection*{Problem 17}

Let $f: X \to Y$ be a function. Prove that $f(A\cup B) = f(A) \cup f(B)$ for all
$A, B \in X$.

Let $y \in Y$ and $x \in X$ such that $f(x) = y$ and $y \in f(A\cup B)$. $y \in f(A\cup B)$
if and only if $x \in A \cup B$. Equivalently we can say $x \in A$ or $x \in B$,
which is true if
and only if $y \in f(A)$ or $y \in f(B)$. By definition, this means $y \in f(A) \cup f(B)$.
Since $y \in f(A\cup B) \iff y \in f(A) \cup f(B)$, $f(A \cup B) = f(A) \cup f(B)$ for
all $A, B \in X$.

\subsection*{Problem 18}

Let $f:X\to Y$ be a function.

a) Prove that $f(A \cap B) \subseteq f(A) \cap f(B)$ for all sets $A, B \subseteq X$
but that equality does not hold in general. 

Let $y \in f(A \cap B)$. There must exist at least one $x \in A \cap B$ such that $f(x)=y$. 
By definition $x \in A$ and $x \in B$, so $y \in f(A)$ and $y \in f(B)$.
Since this is true for 
any $y \in f(A \cap B)$, it must be that $f(A \cap B) \subseteq f(A) \cap f(B)$.

We can show a case where $f(A \cap B) \subsetneq f(A) \cap f(B)$.

Take $X = \{ 1, 2, 3 \}, Y = \{ 8, 9\}$ and $A, B \subseteq X$ where $A = \{1, 2\}$
and $B = \{1, 3\}$. Define $f:X\to Y$ such that $f(1) = 9, f(2) = 8, f(3) = 8$.
Now we find that $(A \cap B) = \{1\}, f(A \cap B) = \{9\}, f(A) = \{8, 9\},
f(B) = \{8, 9\}$, and $(f(A) \cap f(B)) = \{8, 9\}$. Note that $\{9\} \subsetneq \{8, 9\}$
in this case, but it still satisfies the property we proved above. 

b) Prove that if $f$ is injective, $f(A \cap B) = f(A) \cap f(B)$.

We already showed that for general $f$, $f(A \cap B) \subseteq f(A) \cap f(B)$. So we only
need to prove that $f(A) \cap f(B) \subseteq f(A \cap B)$ in order to conclude that
$f(A \cap B) \subseteq f(A) \cap f(B)$. Let $y \in f(A) \cap f(B)$, which means that 
$y \in f(A)$ and $y \in f(B)$. Since $f$ is injective, there exists a unique $x$ such
that $f(x) = y$. This means that $x$ must be in both $A$ and $B$. Thus, $x \in A\cap B$, 
so $y \in f(A \cap B)$. Since all $y \in f(A) \cap f(B)$ is also in $f(A \cap B)$, 
$f(A) \cap f(B) \subseteq f(A \cap B)$ and thus $f(A \cap B) = f(A) \cap f(B)$.

\subsection*{Problem 19}

Let $f : X \to Y$ be a function and let $A, B \subseteq Y$.

\textbf{Show that} $f^{-1}(A \cup B) = f^{-1}(A) \cup f^{-1}(B)$:

For all $x \in X$,

$$
\begin{array}{rl}
x \in f^{-1}(A \cup B) &\iff f(x) \in A \cup B \\
&\iff f(x) \in A \lor f(x) \in B \\
&\iff x \in f^{-1}(A) \lor x \in f^{-1}(B) \\
&\iff x \in f^{-1}(A) \cup f^{-1}(B).
\end{array}
$$

Since $x \in f^{-1}(A \cup B) \iff x \in f^{-1}(A) \cup f^{-1}(B)$,
we conclude that $f^{-1}(A \cup B) = f^{-1}(A) \cup f^{-1}(B)$.


\textbf{Show that} $f^{-1}(A \cap B) = f^{-1}(A) \cap f^{-1}(B)$:

For all $x \in X$,

$$
\begin{array}{rl}
x \in f^{-1}(A \cap B) &\iff f(x) \in A \cap B \\
&\iff f(x) \in A \land f(x) \in B \\
&\iff x \in f^{-1}(A) \land x \in f^{-1}(B) \\
&\iff x \in f^{-1}(A) \cap f^{-1}(B).
\end{array}
$$

Since $x \in f^{-1}(A \cap B) \iff x \in f^{-1}(A) \cap f^{-1}(B)$,
we conclude that $f^{-1}(A \cap B) = f^{-1}(A) \cap f^{-1}(B)$.


\textbf{Show that} $f^{-1}(A \setminus B) = f^{-1}(A) \setminus f^{-1}(B)$:

For all $x \in X$,

$$
\begin{array}{rl}
x \in f^{-1}(A \setminus B) &\iff f(x) \in A \setminus B \\
&\iff f(x) \in A \land f(x) \notin B \\
&\iff x \in f^{-1}(A) \land x \notin f^{-1}(B) \\
&\iff x \in f^{-1}(A) \setminus f^{-1}(B).
\end{array}
$$

Since $x \in f^{-1}(A \setminus B) \iff x \in f^{-1}(A) \setminus f^{-1}(B)$,
we conclude that $f^{-1}(A \setminus B) = f^{-1}(A) \setminus f^{-1}(B)$.

\end{document}
