\documentclass[11pt]{article}
\usepackage{amsfonts}
\usepackage{amsmath}
\usepackage{amssymb}
\usepackage{enumitem}

\newcommand{\N}{\mathbb{N}}
\newcommand{\Z}{\mathbb{Z}}
\newcommand{\Q}{\mathbb{Q}}
\newcommand{\R}{\mathbb{R}}
\renewcommand{\P}{{\cal P}}

\pagestyle{empty}


\begin{document}

\setlength{\parindent}{0pt}
\setlength{\parskip}{9pt}


\section*{Math 351: Homework 7  Due Friday November 2}
\subsection*{Jack Ellert-Beck}

\bigskip

\subsection*{Problem 1}

Suppose $I$ is a bounded closed interval and $f:I\to\R$ is continuous on
$I$. We will prove that $f(I)$ is bounded. Since $I$ is a bounded closed
interval on $R$, it is a compact subset of $R$. By the Extreme Value
Theorem, $f$ has a maximum and a minimum on $I$. This means that 
$f(I)$ has a maximum and minimum value, and thus it is bounded.

Suppose $I$ is a bounded open interval and $f:I\to\R$ is continuous on $I$.
We will show a counterexample to disprove that $f(I)$ is bounded.
Let $I$ be $(0,1)$ and define $f(x) = \frac{1}{x}$. $I$ is a bounded open
interval and we have previously shown $f$ is continuous on $I$. However,
$f(x)\to\infty$ as $x\to0$, so $f(I) = (1,\infty)$ is not bounded.

Suppose $I$ is a bounded open interval and $f:I\to\R$ is uniformly
continuous on $I$. We will prove that $f(I)$ is bounded. It will
suffice to show that $f(I)$ must be bounded from above, since to show
that $f(I)$ is bounded from below is the same as showing that
$-f(I)$ is bounded from above.
So, assume for contradiction that $f(I)$ is not bounded from above.
This means that $\forall M\in\R\ \exists e\in f(I)$ such that $e>M$.
Note that for any such $e$ there is some $x\in I$ where $f(x)=e$.
Now, by the definition of uniform continuity,
$\forall\varepsilon>0\ \exists\delta>0$ such that
$x,y\in I$ and $|x-y|<\delta\implies|f(x)-f(y)|<\varepsilon$.
Pick any $\varepsilon$ and fix it. Now find a corresponding $\delta$.
Now, let $a,b$ be the endpoints of $I$ such that $I=(a,b)$. Note that
at least one of $f((a,\frac{a+b}{2}])$ or $f([\frac{a+b}{2},b))$ is
not bounded. (If both subintervals were bounded, then $\sup I$ would
be a real number
and be equal to the larger of the two supremums of the subintervals.)
If the image of the former interval under $f$ is unbounded,
let $a_1=a$ and $b_1=\frac{a+b}{2}$.
If not, then let $a_1=\frac{a+b}{2}$ and $b_1=b$. Now let $I_1=(a_1,b_1)$.
Again, we can divide $I_1$ in half and choose whichever half has an unbounded
image under $f$
to be $I_2$. Continuing this pattern, note that $(a_n-b_n)=\frac{a-b}{2^n}\to0$
as $n\to\infty$, and that for all $I_n$, $f(I_n)$ is unbounded.
Thus there is some $I_n$ where $(a_n-b_n)<\delta$. Now pick $y=a_n$ and let
$M=f(y)+\varepsilon$. By the fact that $I_n$ is unbounded, there is some
$x\in I_n$ such that $f(x)>f(y)+\varepsilon$. But, $x$ is in $I_n$,
so we have found a case where $|x-y|<\delta$ and 
$|f(x)-f(y)|>\varepsilon$, a contradiction. So $f(I)$ must be bounded from
above, and following the same argument with $-f(I)$ shows that
$f(I)$ is also bounded from below, so $f(I)$ is bounded.

Suppose $A$ is a countable union of bounded open intervals and $f:A\to\R$
is uniformly continuous on $A$. We will provide a counterexample to show
that $f(A)$ need not be bounded. Let
$A=\ldots\cup (-2, -1) \cup (0, 1) \cup (2, 3) \cup\ldots$ and $f(x)=x$.
We can show that $f$ is uniformly continuous on $\R$. Pick any $\varepsilon>0$.
If we choose $\delta=\frac{\varepsilon}{2}$, we can see that $|x-y|<\delta
\implies |f(x)-f(y)|=|x-y|<\frac{\varepsilon}{2}<\varepsilon$. Since
$f$ is uniformly continuous on $\R$, it is uniformly continuous on
$A\subseteq\R$. However, we can see that $f(I)$ is not bounded. If we
choose any $M>0$, we know that $f(M+1)=M+1>M$. Note that
at least one of $M+1, M+2, M+2.1$, and $M+3.1$ will be in $A$. Thus we
can always find an $x\in A$ where $f(x)>M$. So in this case, 
$f$ is uniformly continuous on a countable union of bounded open intervals
and $f(I)$ is not bounded.

\subsection*{Problem 2}

Suppose $a_n$ and $b_n$ are Cauchy sequences.

Prove or disprove: $|a_n-b_n|$ is Cauchy.
Since both sequences are Cauchy, we know that $a_n$ converges to
some $A\in\R$ and $b_n$ converges to a value $B\in\R$.
So, $\forall\varepsilon>0\ \exists N_1>0$ such that
$n>N_1\implies |a_n-A|<\varepsilon$ and $n>N_2\implies|b_n-B|<\varepsilon$.
Pick an $\varepsilon$ and fix it. Find any $N_1,N_2$ corresponding to
$\frac{\varepsilon-(A-B)}{2}$ and set $N>\max \{N_1,N_2\}$. Now we have
$|a_n-b_n|=|a_n-A-b_n+B+A-B|=|(a_n-A)+(B-b_n)+(A-B)|<\varepsilon-(A-B)+(A-B)
=\varepsilon$. Thus $|a_n-b_n|$ converges and is therefore Cauchy.

Prove or disprove: $(-1)^na_n$ is Cauchy.
Let $a_n=1$ This clearly converges and thus is Cauchy. 
However, $(-1)^na_n$ does not converge. To see why, pick
$\varepsilon=\frac{1}{2}$. For any choice of $N$,
$|a_{N+2}-a_{N+1}|=2>\varepsilon$. Thus $(-1)^na_n$ is not Cauchy.

Prove or disprove: if $a_n\neq0$ for all $n$, then $\frac{1}{a_n}$ is Cauchy.
Let $a_n=\frac{1}{2^n}$. This sequence converges to 0, and for all
$n$, $a_n>0$. So $a_n$ is Cauchy. However, $\frac{1}{a_n}=2^n\to\infty$.
So, $\frac{1}{a_n}$ does not converge, and this counterexample disproves
the proposition.

Prove or disprove: if $a_n>0.001$ for all $n$, then $\frac{1}{a_n}$ is Cauchy.
If $a_n>0.001$ for all $n$ and $a_n$ is Cauchy, then $a_n$ converges
to some $L>0.001$. Hence $\frac{1}{a_n}<1000$ and, by the algebraic properties
of limits, $\frac{1}{a_n}$ converges to $\frac{1}{L}$, and we conclude that
$\frac{1}{a_n}$ is Cauchy.

\subsection*{Problem 3}

A function $f:\R\to\R$ is monotone if $x\leq y$ makes $f(x)\leq f(y)$.
Show that a monotone function can have at most a countable number of
points of discontinuity.

First, consider the function $f(x)=\lfloor x\rfloor$ as an example of
a monotone function with a countably infinite number of points of
discontinuity. So, clearly a monotone function can have an infinite number of
discontinuities.

In order to show this for $\R$, we will first show that $f$ must have
at most a countable number of discontinuities on the interval
$[a,b]$.
At a point of discontinuity $c\in[a,b]$ we have that
$\exists\varepsilon>0$ such that $\forall\delta>0$,
$|x-c|<\delta$ and $|f(x)-f(c)|>\varepsilon$. Note that every time we have
a discontinuity, the value of $f$ can only jump up, because $f$ is
monotone and also defined on all of $[a,b]$.
Now, pick a discontinuity point $c$ and find a corresponding
$\varepsilon$ as per the definition. There must be a finite number of jumps
of size $\varepsilon$ on $[a,b]$ because the set $f([a,b])$ is bounded.
We can find at most a finite set of discontinuities bigger than
$\varepsilon$ for
any value of $\varepsilon$. So, the set of all discontinuous points on
$[a,b]$ is a union of finite sets, so it is at most countable.
We can cover $\R$ with a countable union of bounded closed sets,
each containing a countable number of discontinuities, so the total
set of all discontinuities is a countable union of countable sets,
which itself must be countable.


\subsection*{Section 5.1}

\begin{enumerate}
\setcounter{enumi}{3}

\item{Prove that $f$ is continuous on $[-1,1]$, but $f$ is not differentiable
at $x=0$ where $f:[-1,1]\to\R$ is defined by:}
\[f(x)=\begin{cases}
        x\sin(1/x)  & x\neq0\\
        0           & x=0
\end{cases}
\]

To show that $f$ is continuous on $[-1,1]$, we can show that for any sequence
$(a_n)$ in $[-1,1]$, $a_n\to x\implies f(a_n)\to f(x)$.
Pick any sequence $(a_n)$ in the domain and call its limit $L$. Then
$\lim_{n\to\infty}f(a_n)=\lim_{n\to\infty}a_n\sin(1/a_n)=
(\lim_{n\to\infty}a_n)\cdot(\lim_{n\to\infty}\sin(1/a_n))=
L\cdot\lim_{n\to\infty}\sin(1/a_n)$. Since $\sin(x)$ is continuous and
$1/x$ is continuous, their composition is continuous, which means that
$\lim_{n\to\infty}\sin(1/a_n)=\sin\left(\frac{1}{\lim_{n\to\infty}a_n}\right)=
\sin(1/L)$. So, $\lim_{n\to\infty}f(a_n)=L\cdot(1/L)=f(L)$ and we conclude
that $f(x)$ is continuous.

However, we can show that $f$ is not differentiable at 0 by showing that
$f$ is not locally linear near 0. We want to try to write
$f(x)=0+M\cdot x+o(x)$ for all $x\in B(0,\varepsilon)$ and $\varepsilon>0$.
We know that $f(x)=x$ is continuous, so we can write
$f(x)=x\sin(1/x)=(M_1\cdot x+o(x))(\sin(1/x))=M_1\cdot x\sin(1/x)+o(x)\sin(1/x)
=M_1\cdot x\sin(1/x)+o(x)$. However $\sin(1/x)$ is not continuous and thus
not differentiable at 0, so it is not locally linear near 0, so
we cannot simplify this expression to the desired form. Thus,
$f$ is not differentiable at 0.


\item{Prove that $g$ is differentiable on $(-1,1)$ but $g'$ is not
continuous at $x=0$ where $g$ is defined by:}
\[f(x)=\begin{cases}
        x^2\sin(1/x)    & x\neq0\\
        0               & x=0
\end{cases}
\]

Ran out of time to figure this one out. 
I'm confused because, what does it mean for $g'$ to not be continuous at a 
point if it is defined at that point? I would assume it means that
the left-sided limit and the right-sided limit of $g'$ are different.
But doesn't that mean $g$ is not differentiable at that point? I
had trouble figuring out what I can do to prove anything here.

\item{Give an example of a continuous invertible function on $(0,1)$ that
is not differentiable.}

Let $f:(0,1)\to\R$ be defined by
\[f(x)=\begin{cases}
        x & x\leq\frac{1}{2}\\
        2x-\frac{1}{2} & x>\frac{1}{2}
\end{cases}
\]

Note that at $x=\frac{1}{2}$ the left-sided derivative is 1, but the
right-sided derivative is 2, so $f$ is not differentiable. But the inverse
of $f$ is well-defined:
\[f(x)=\begin{cases}
        x & x\leq\frac{1}{2}\\
        \frac{1}{2}x+\frac{1}{4} & x>\frac{1}{2}
\end{cases}
\]
\item{Prove that a continuous invertible function defined on a closed
or open interval is strictly monotone.}

Let $f$ be a continuous invertible function and assume for contradiction that
$f$ is not strictly monotone. This means that there exist some
$a<b<c$ in the domain such that $f(a)\leq f(b)$ and $f(c)\leq f(b)$ or where
$f(a)\geq f(b)$ and $f(c)\geq f(b)$. We can assume the former case is true
without losing generality. In addition, we consider the case where
$f(a)\leq f(c)$. By the Intermediate Value Theorem, there exists some
$a'\in[a,b]$ such that $f(a')=f(c)$ since $f(c)\in[f(a),f(b)]$.
But, if $f(a')=f(c)$ then $f$ is not one-to-one, so $f$ is not invertible,
a contradiction. So continuous invertible functions must be
strictly monotone.

\item{Let $n\in\N$. Construct a function that is $n$ times differentiable
on an interval but fails to be $n+1$ times differentiable at a point in
the interval.}

Define $f:[0,\infty)\to\R$ as $f(x)=x^{\frac{1}{2}+n}$. We can differentiate
$f$ $n$ times on $[0,\infty)$, and
$f^N(x)=\left(\Pi_{i=1}^{n}(\frac{1}{2}+i)\right)x^{\frac{1}{2}}$.
However, $x^\frac{1}{2}$ is not differentiable at $x=0$.

\end{enumerate}

\subsection*{Section 5.2}

\begin{enumerate}
\item{Let $f:[a,b]\to\R$ be a function. Suppose that $f$ is continuous on
$[a,b]$ and differentiable on $(a,b)$. Show that if $f'(x)\neq0$ for all
$x\in(a,b)$, then $f$ is one-to-one.}

We prove this by proving the contrapositive. Assume $f$ is not one-to-one,
which means that $\exists x<y\in(a,b)$ such that $f(x)=f(y)$.
By Rolle's Theorem, there must be a $c\in(x,y)$ such that $f'(c)=0$.
So we have shown the contrapositive, and thus the original statement
is true.

\item{Let $f,g:[a,b]\to\R$ be functions that are continuous on $[a,b]$ and
differentiable on $(a,b)$. Show that if $f'(x)=g'(x)$ for all $x\in(a,b)$
then there exists $c\in\R$ such that $f(x)=g(x)+c$ for all $x\in[a,b]$.}

Since $f'(a)=g'(a)$ then $f$ and $g$ are locally linear near $a$ with the
same slope $M$. This means that $f(x)=f(a)+M(x-a)+o(x-a)$ and
$g(x)=g(a)+M(x-a)+o(x-a)$. Now subtract $g$ from $f$:
$f(x)-g(x)=f(a)+M(x-a)+o(x-a)-g(a)-M(x-a)-o(x-a)=
(f(a)-g(a))+o(x-a)-o(x-a)$ Now, as $x\to a$,
$f(x)-g(x)=(f(a)-g(a))\implies f(x)=g(x)+(f(a)-g(a))$. Note that
$f(a)-g(a)$ is a constant value, so we have proven the statement true.

\item{Let $f,g:[a,b]\to\R$ be functions that are continuous on $[a,b]$
and differentiable on $(a,b)$ such that $f(a)=g(a)$. Show that if
$f'(x)\leq g'(x)$ for all $x\in(a,b)$ then $f(x)\leq g(x)$ for all
$x\in(a,b)$.}

We will prove this by the contrapositive. Assume that $f(a)=g(a)$ and that
$\exists c\in[a,b]$ such that $f(c)>g(c)$. Now define $h(x)=g(x)-f(x)$
and note that $h(a)=0$ and $h(c)=d$ for some value of $d<0$. 
By the Mean Value Theorem, there is some $c'\in(a,c)$ such that
$h'(c')=\frac{d-0}{c-a}<0$. Note that $h'(c')=g'(c')-f'(c')$, and we have
$g'(c')-f'(c')<0\implies g'(c')<f'(c')$. We have shown that if
$f(x)>g(x)$ for any $x$ then there must be some value where
$f'(x)>g'(x)$. The contrapositive has been shown to be true, so the original
proposition is true.


\setcounter{enumi}{7}
\item{Let $f:(a,b)\to\R$ be a differentiable function. Prove that if
$f'(x)\geq 0$ for all $x\in(a,b)$ then $f$ is an increasing function on
$(a,b)$. Furthermore, if $f'(x)>0$ for all $x\in(a,b)$, then $f$ is a
a strictly increasing function on $(a,b)$.}

We will prove this by contraposition. Assume that $\exists x<y\in(a,b)$
such that $f(x)>f(y)$, and we will show that there is some $c$ where
$f'(c)<0$. By the Mean Value Theorem, we can see that $\exists c\in(x,y)$
such that $f'(c)=\frac{f(y)-f(x)}{y-x}$. Since $y>x$ and $f(y)<f(x)$,
$f'(c)$ is negative. So we have shown the contrapositive to be true,
and we conclude that the original statement is true.

\item{Prove that the function $g:(-1,1)\to\R$ given by $g(x)=\sqrt[3]{x}$
is not differentiable at 0.}

The derivative of $g(x)=\sqrt[3](x)=x^{1/3}$ on $(-1,0)\cup(0,1)$ is
$g'(x)=\frac{1}{3}x^{-2/3}$. This function approaches infinity
as $x\to0$ from either side, so $g'$ is not defined at 0.



\end{enumerate}

\end{document}
