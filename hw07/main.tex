\documentclass[11pt]{article}
\usepackage{amsfonts}
\usepackage{amsmath}
\usepackage{amssymb}
\usepackage{enumitem}

\newcommand{\N}{\mathbb{N}}
\newcommand{\Z}{\mathbb{Z}}
\newcommand{\Q}{\mathbb{Q}}
\newcommand{\R}{\mathbb{R}}
\renewcommand{\P}{{\cal P}}

\pagestyle{empty}


\begin{document}

\setlength{\parindent}{0pt}
\setlength{\parskip}{9pt}


\section*{Math 351: Homework 7  Due Friday November 2}
\subsection*{Jack Ellert-Beck}

\bigskip

\subsection*{Problem 1}

Suppose $I$ is a bounded closed interval and $f:I\to\R$ is continuous on
$I$. We will prove that $f(I)$ is bounded. Since $I$ is a bounded closed
interval on $R$, it is a compact subset of $R$. By the Extreme Value
Theorem, $f$ has a maximum and a minimum on $I$. This means that 
$f(I)$ has a maximum and minimum value, and thus it is bounded.

Suppose $I$ is a bounded open interval and $f:I\to\R$ is continuous on $I$.
We will show a counterexample to disprove that $f(I)$ is bounded.
Let $I$ be $(0,1)$ and define $f(x) = \frac{1}{x}$. $I$ is a bounded open
interval and we have previously shown $f$ is continuous on $I$. However,
$f(x)\to\infty$ as $x\to0$, so $f(I) = (1,\infty)$ is not bounded.

Suppose $I$ is a bounded open interval and $f:I\to\R$ is uniformly
continuous on $I$. We will prove that $f(I)$ is bounded. It will
suffice to show that $f(I)$ must be bounded from above, since to show
that $f(I)$ is bounded from below is the same as showing that
$-f(I)$ is bounded from above.
So, assume for contradiction that $f(I)$ is not bounded from above.
This means that $\forall M\in\R\ \exists e\in f(I)$ such that $e>M$.
Note that for any such $e$ there is some $x\in I$ where $f(x)=e$.
Now, by the definition of uniform continuity,
$\forall\varepsilon>0\ \exists\delta>0$ such that
$x,y\in I$ and $|x-y|<\delta\implies|f(x)-f(y)|<\varepsilon$.
Pick any $\varepsilon$ and fix it. Now find a corresponding $\delta$.
Now, let $a,b$ be the endpoints of $I$ such that $I=(a,b)$. Note that
at least one of $f((a,\frac{a+b}{2}])$ or $f([\frac{a+b}{2},b))$ is
not bounded. (If both subintervals were bounded, then $\sup I$ would
be a real number
and be equal to the larger of the two supremums of the subintervals.)
If the image of the former interval under $f$ is unbounded,
let $a_1=a$ and $b_1=\frac{a+b}{2}$.
If not, then let $a_1=\frac{a+b}{2}$ and $b_1=b$. Now let $I_1=(a_1,b_1)$.
Again, we can divide $I_1$ in half and choose whichever half has an unbounded
image under $f$
to be $I_2$. Continuing this pattern, note that $(a_n-b_n)=\frac{a-b}{2^n}\to0$
as $n\to\infty$, and that for all $I_n$, $f(I_n)$ is unbounded.
Thus there is some $I_n$ where $(a_n-b_n)<\delta$. Now pick $y=a_n$ and let
$M=f(y)+\varepsilon$. By the fact that $I_n$ is unbounded, there is some
$x\in I_n$ such that $f(x)>f(y)+\varepsilon$. But, $x$ is in $I_n$,
so we have found a case where $|x-y|<\delta$ and 
$|f(x)-f(y)|>\varepsilon$, a contradiction. So $f(I)$ must be bounded from
above, and following the same argument with $-f(I)$ shows that
$f(I)$ is also bounded from below, so $f(I)$ is bounded.

Suppose $A$ is a countable union of bounded open intervals and $f:A\to\R$
is uniformly continuous on $A$. We will provide a counterexample to show
that $f(A)$ need not be bounded. Let
$A=\ldots\cup (-2, -1) \cup (0, 1) \cup (2, 3) \cup\ldots$ and $f(x)=x$.
We can show that $f$ is uniformly continuous on $\R$. Pick any $\varepsilon>0$.
If we choose $\delta=\frac{\varepsilon}{2}$, we can see that $|x-y|<\delta
\implies |f(x)-f(y)|=|x-y|<\frac{\varepsilon}{2}<\varepsilon$. Since
$f$ is uniformly continuous on $\R$, it is uniformly continuous on
$A\subseteq\R$. However, we can see that $f(I)$ is not bounded. If we
choose any $M>0$, we know that $f(M+1)=M+1>M$. Note that
at least one of $M+1, M+2, M+2.1$, and $M+3.1$ will be in $A$. Thus we
can always find an $x\in A$ where $f(x)>M$. So in this case, 
$f$ is uniformly continuous on a countable union of bounded open intervals
and $f(I)$ is not bounded.

\subsection*{Problem 2}

Suppose $a_n$ and $b_n$ are Cauchy sequences.

Prove or disprove: $|a_n-b_n|$ is Cauchy.
Since both sequences are Cauchy, we know that $a_n$ converges to
some $A\in\R$ and $b_n$ converges to a value $B\in\R$.
So, $\forall\varepsilon>0\ \exists N_1>0$ such that
$n>N_1\implies |a_n-A|<\varepsilon$ and $n>N_2\implies|b_n-B|<\varepsilon$.
Pick an $\varepsilon$ and fix it. Find any $N_1,N_2$ corresponding to
$\frac{\varepsilon-(A-B)}{2}$ and set $N>\max \{N_1,N_2\}$. Now we have
$|a_n-b_n|=|a_n-A-b_n+B+A-B|=|(a_n-A)+(B-b_n)+(A-B)|<\varepsilon-(A-B)+(A-B)
=\varepsilon$. Thus $|a_n-b_n|$ converges and is therefore Cauchy.

Prove or disprove: $(-1)^na_n$ is Cauchy.
Let $a_n=1$ This clearly converges and thus is Cauchy. 
However, $(-1)^na_n$ does not converge. To see why, pick
$\varepsilon=\frac{1}{2}$. For any choice of $N$,
$|a_{N+2}-a_{N+1}|=2>\varepsilon$. Thus $(-1)^na_n$ is not Cauchy.

Prove or disprove: if $a_n\neq0$ for all $n$, then $\frac{1}{a_n}$ is Cauchy.
Let $a_n=\frac{1}{2^n}$. This sequence converges to 0, and for all
$n$, $a_n>0$. So $a_n$ is Cauchy. However, $\frac{1}{a_n}=2^n\to\infty$.
So, $\frac{1}{a_n}$ does not converge, and this counterexample disproves
the proposition.

Prove or disprove: if $a_n>0.001$ for all $n$, then $\frac{1}{a_n}$ is Cauchy.
If $a_n>0.001$ for all $n$ and $a_n$ is Cauchy, then $a_n$ converges
to some $L>0.001$. Hence $\frac{1}{a_n}<1000$ and, by the algebraic properties
of limits, $\frac{1}{a_n}$ converges to $\frac{1}{L}$, and we conclude that
$\frac{1}{a_n}$ is Cauchy.

\subsection*{Problem 3}

A function $f:\R\to\R$ is monotone if $x\leq y$ makes $f(x)\leq f(y)$.
Show that a monotone function can have at most a countable number of
points of discontinuity.



\subsection*{Section 5.1}

\begin{enumerate}
\setcounter{enumi}{3}

\item{Prove a thing}

\item{Prove a thing}

\item{Give an example}

\item{Prove a thing}

\item{Construct a function}

\end{enumerate}

\subsection*{Section 5.2}

\begin{enumerate}
\item{Who knows}

\item{not me}

\item{yikes}

\setcounter{enumi}{7}
\item{8 really}

\item{9 huh}

\end{enumerate}

\end{document}
