\documentclass[11pt]{article}
\usepackage{amsfonts}
\usepackage{amsmath}

\newcommand{\N}{\mathbb{N}}
\newcommand{\Z}{\mathbb{Z}}
\newcommand{\Q}{\mathbb{Q}}
\newcommand{\R}{\mathbb{R}}
\renewcommand{\P}{{\cal P}}

\pagestyle{empty}


\begin{document}

\setlength{\parindent}{0pt}
\setlength{\parskip}{9pt}


\section*{Math 351: Homework 2 (Due September 21)}
\section*{Name: Jack Ellert-Beck}

\bigskip

\subsection*{Section 0.5}
\subsection*{Problem 7}

We will show by induction that for all integers $n \geq 1, n < 2^n$.

For the base case, let $n=1$. $2^n = 2^1 = 2 > 1$, so the property holds.

Now we will show that if $n < 2^n$ then $n+1 < 2^{n+1}$. We can add one to both sides
of the first inequality: 
$n+1 < 2^n + 1 < 2^n + 2^n < 2^n\cdot 2 = 2^{n+1}$. This completes the inductive step.
Thus, by mathematical induction, the property is true for all $n \geq 1$.

\subsection*{Problem 8}

Let $\{F_n\}$ be the Fibonacci sequence. We will prove that for all $n \in \N$,
\[ \sum^n_{i=0}F_i = F_{n+2}-1. \]

We proceed by induction. Let $n=1$ and recall that $F_1=1$ and $F_3=2$.
We see that $F_1 = 1 = 2 - 1 = F_3-1$.

For the inductive step we assume $ \sum^n_{i=0}F_i = F_{n+2}-1 $, and we want
to show that $ \sum^{n+1}_{i=0}F_i = F_{n+3}-1 $. In addition,
recall that the definition of $\{F_n\}$ includes the statement 
$F_n = F_{n-1} + F_{n+2}$. Consider the following equalities:

\[
\begin{array}{rl}
\sum^{n+1}_{i=0}F_i &= \sum^n_{i=0}F_i + F_{n+1} \\
&= F_{n+2} - 1 + F_{n+1} \\
&= (F_{n+2} + F_{n+1}) - 1 \\
&= F_{n+3} - 1.
\end{array}
\]
By mathematical induction, this property must hold for all $n \in \N$.

\subsection*{Problem 9}
We will show that for all integers $n \geq 1$,
\[ \sum^n_{i=1}\frac{1}{2^i} = 1-\frac{1}{2^n}. \]

We proceed by induction. Consider the base case where $n=1$. We get
$\frac{1}{2^1} = \frac{1}{2} = 1 - \frac{1}{2} = 1 - \frac{1}{2^1}$, so the
base case is true.

For the inductive step we assume $\sum^n_{i=1}\frac{1}{2^i} = 1-\frac{1}{2^n}$
and we want to show that $\sum^{n+1}_{i=1}\frac{1}{2^i} = 1-\frac{1}{2^{n+1}}$.
Note that the following equalities hold:

\[
\def\arraystretch{1.5}
\begin{array}{rl}
\sum^{n+1}_{i=1}\frac{1}{2^i} &= \sum^{n}_{i=1}\frac{1}{2^i} + \frac{1}{2^{n+1}} \\
&= 1 - \frac{1}{2^n} + \frac{1}{2^{n+1}} \\
&= 1 - \frac{2^{n+1} - 2^n}{2^n\cdot 2^{n+1}} \\
&= 1 - \frac{2\cdot 2^n - 2^n}{2^n\cdot 2^{n+1}} \\
&= 1 - \frac{2^n}{2^n\cdot 2^{n+1}} \\
&= 1 - \frac{1}{2^{n+1}}.
\end{array}
\]
So, by mathematical induction, this is true for all $n \geq 1$.

\subsection*{Problem 11}

We will show that for all integers $n \geq 1$,
\[ \sum^n_{i=1}i^2= \frac{n(n+1)(2n+1)}{6}. \]

We can show this by induction. We let $n=1$ for the base case, and we see that
$1^2 = 1 = \frac{6}{6} = \frac{1\cdot 2\cdot 3}{6}$.

For the inductive step we assume $\sum^n_{i=1}i^2= \frac{n(n+1)(2n+1)}{6}$
and we want to show that 
$\sum^{n+1}_{i=1}i^2= \frac{(n+1)(n+2)(2(n+1)+1)}{6}=\frac{(n+1)(n+2)(2n+3)}{6} $.
Consider the following equalities:

\[
\def\arraystretch{1.5}
\begin{array}{rl}
\sum^{n+1}_{i=1}i^2 &= \sum^{n}_{i=1}i^2 + (n+1)^2 \\
&= \frac{n(n+1)(2n+1)}{6} + (n+1)^2 \\
&= \frac{n(n+1)(2n+1) + 6(n+1)^2}{6} \\
&= \frac{(n+1)(n(2n+1) + 6(n+1))}{6} \\
&= \frac{(n+1)(2n^2+7n+6)}{6} \\
&= \frac{(n+1)(n+2)(2n+3)}{6}.
\end{array}
\]
By mathematical induction, this property is true for all $n \geq 1$.

\subsection*{Section 1.3}
\subsection*{Problem 7}

Recall that for sets $A_1$ and $A_2$, the Cartesian product is defined as  
$A \times A_1 = \{ (a_1, a_2) \mid a_1 \in A_1, a_2 \in A_2\} $.
Define the Cartesian product of $k$ sets 
$A_1 \times A_2 \times \ldots \times A_k =
\{ (a_1, a_2, \ldots, a_k) \mid a_1 \in A_1, a_2 \in A_2, \ldots, a_k \in A_k\}$.

Using this definition we can prove that if the sets $A_1, \ldots, A_k$ are 
countable, then their Cartesian product is also countable. We proceed by induction.
In the case where $k = 2$, we have the Cartesian product $A_1 \times A_2$, which
is countable by Proposition 1.3.8.
For the inductive step, assume that $A_1,\ldots,A_{k+1}$ are countable and that
$A_1 \times \ldots \times A_k$ is countable. The Cartesian product
$A_1 \times \ldots \times A_{k+1}$ can be written as
$(A_1 \times \ldots \times A_k) \times A_{k+1}$. The set $A_1 \times \ldots \times A_k$
is assumed to be countable, as is $A_{k+1}$, so their Cartesian product 
$(A_1 \times \ldots \times A_k) \times A_{k+1}$ is countable by 
Proposition 1.3.8. So, by induction, the Cartesian product of $k$
countable sets is itself countable.

\subsection*{Argue that the set of finite sequences of 0's and 1's is countable}

Consider the set of finite sequences of 0's and 1's, which we will call $S$.
We can interpret elements of $S$ as binary representations of natural numbers.
This is actually a surjection from $\N_0$ onto $S$, as any 
sequence in $S$ can be interpreted as a natural number. Since we have mapped
$\N_0$, which is countable, onto $S$, $S$ must also be countable.

\subsection*{Argue that the set of infinite sequences of 0's and 1's is not countable}

Call the set of infinite sequences of 0's and 1's $S'$. Cantor's diagonalization
argument can be applied to $S'$. Assume for contradiction that $S'$ is countable
and that as a result there exists some ordered list of each element of $S'$.
We can create
a new sequence $a$ whose first term is different from the first term of the 
first sequence in $S'$, whose second term is different from the second term
of the second sequence in $S'$, and so on. The sequence $a$ differs from every
sequence in $S'$ in at least one position, so it cannot be on the list even though
it is a member of $S'$. We have encountered a contradicton, and thus 
$S'$ is not countable.

\subsection*{Problem 13}

We will show that the set of algebraic numbers is countable by showing first that
the set of polynomials, $P$, is countable. Note that any algebraic number $a$ 
is by definition the root of some polynomial $p \in P$, or, in other words,
there exists an injection from the algebraic numbers into $P$ (the relation $f$
where $f(a) = p$ only if $a$ is a root of $p$), which means
that the cardinality of the algebraic numbers is less than or equal to that of $P$.
So, if we show that $P$ is countable, we can conclude that the algebraic numbers
are countable.

Let $P_n$ be the set of polynomials of degree $k$: specifically
$P_n = \{ a_nx^n + a_{n-1}x^{n-1}+\ldots+a_1x+a_0 \mid a_n,\ldots,a_0\in \Z \}$.
In addition, consider the Cartesian product $\Z^n$ to be the Cartesian product of
$\Z$ $n$ times, as defined in Problem 7. We can define the relation 
$f: \Z^n \to P_n$ such that, if $b = (b_0,\ldots,b_n) \in \Z^n, p \in P_n$, then
$f(b) = p$ only if $p$'s coefficient $a_k$ = $b_k$ for all $0\leq k \leq n$. 
Every $p \in P_n$ has coefficients from $\Z$, so every $p = f(b)$ for some 
$b \in \Z^n$. Thus, $f$ is surjective. Since $\Z$ is countable, $\Z^n$ is countable
by Problem 7, and since $f$ is a surjection onto $P_n$, each $P_n$ is also countable.

Now, notice that every polynomial $p \in P$ is in $P_n$ for some $n$. This means
that $P = \bigcup_{n=1}^{\infty} P_n$. So, by Theorem 1.3.14, $P$ is itself countable
since it is a countable union of countable sets. By the argument above, we can
conclude that the algebraic numbers are countable.

\subsection*{Section 2.1}
\subsection*{Problem 1}

We will prove that 
$\lim_{n \to \infty} |x_n| = 0 \iff \lim_{n \to \infty} x_n = 0$.
By the definition, $\lim_{n \to \infty} |x_n| = 0 \iff \forall \varepsilon > 0,
\exists N \text{ where } n > N \implies ||x_n|| < \varepsilon$. However,
$||x_n|| = |x_n|$ since $|x_n| \geq 0$. Thus $\exists N$ such that 
$|x_n| < \varepsilon$ for any positive $\varepsilon$, which is the definition of
$\lim_{n \to \infty} x_n = 0$.

\subsection*{Problem 2}

We will use the definition to show that 
$\lim_{n \to \infty} \frac{n}{n+1} = 1$. To satisfy the definition,
then for any $\varepsilon > 0$ we need to find an $N$ such that 
$n > N \implies \left| \frac{n}{n+1} - 1\right| < \varepsilon$.
Choose $N = \frac{1}{\varepsilon} - 1$, so that

\[
\def\arraystretch{1.5}
\begin{array}{rl}
\frac{1}{\varepsilon}-1 &< n\\
\frac{1}{\varepsilon} &< n+1 \\
\frac{1}{n+1} &< \varepsilon \\
\left| \frac{-1}{n+1}\right| &< \varepsilon \\
\left| \frac{n - (n+1)}{n+1}\right| &< \varepsilon \\
\left| \frac{n}{n+1} - 1\right| &< \varepsilon. 
\end{array}
\]
Since we can find an $N$ for any $\varepsilon > 0$ where this inequality holds 
true, by definition $\lim_{n \to \infty} \frac{n}{n+1} = 1$.

\subsection*{Problem 4}

Assume $(x_n)$ is a sequence with $x_n \geq 0$ that converges to $L$.
We will use the definition to show that $\lim_{n\to\infty} \sqrt{x_n} = \sqrt{L}$.

By the definition, we know that $\forall \varepsilon_1 > 0, \exists N_1$ such that
$n > N_1 \implies |x_n - L| < \varepsilon_1$, and we want to show that 
$\forall \varepsilon > 0, \exists N$ where $n > N \implies 
\left| \sqrt{x_n} - L\right| < \varepsilon$. 

First, consider only the case where $L \neq 0$.
Take any $\varepsilon > 0$ and fix it. Now set $\varepsilon_1 = \varepsilon\sqrt{L}$,
which we can do since $x_n \geq 0 \implies L \geq 0$. By the definition stated
above, we know that there exists some $N_1$ where $|x_n-L| < \varepsilon_1$ for all
$n > N_1$. So, when $n > N_1$, (and keeping in mind that $L \neq 0$) we can write 
$\varepsilon\sqrt{L} > |x_n - L| \iff 
\varepsilon > \frac{\left|x_n - L\right|}{\sqrt{L}} \geq 
\frac{\left|x_n - L\right|}{\sqrt{x_n}+\sqrt{L}} =
\left| \frac{(\sqrt{x_n}-\sqrt{L})(\sqrt{x_n}+\sqrt{L})}{\sqrt{x_n}+\sqrt{L}} \right|
= \left|\sqrt{x_n} - \sqrt{L}\right|$.
We showed that there exists an $N$ for any $\varepsilon$, namely, $N = N_1$. 

Now we deal with the case where $L=0$. This simplifies the problem, now
we need to find an $N$ such that when $n > N$, $\left|\sqrt{x_n}\right|< \varepsilon$.
Select some $\varepsilon > 0$ and fix it, and then set $\varepsilon_1 = \varepsilon^2$.
We know that there exists some $N_1$ where for all $n > N_1$,
$|x_n| < \varepsilon_1 = \varepsilon^2$. We take the square root of both sides (and
note that since the left side is always positive, we can discard the negative 
solution on the right)
and see that $\left| \sqrt{x_n} \right| < \varepsilon$ when $n > N_1$. 

Thus we have satisfied the definition
in all cases, so $\lim_{n\to\infty} \sqrt{x_n} = \sqrt{L}$.

\subsection*{Problem 6}
For the following problems, we will use results from other problems, base cases
we proved in class (such as $\lim_{n\to\infty}\frac{1}{n^p}=0$ when $p>0$), 
and Proposition 2.1.13 to break down complicated limits
into solveable problems.

a) $\lim_{n\to\infty}(-1)^n\frac{5}{n}$

From Problem 1, we know that if the sequence of the absolute values converges to 0, then
this sequence will too. So, we evaluate
$\lim_{n\to\infty}\left|(-1)^n\frac{5}{n}\right| = \lim_{n\to\infty}\frac{5}{n}
= \lim_{n\to\infty}5\cdot\lim_{n\to\infty}\frac{1}{n} = 5\cdot0=0$.
Since this converges to 0, so does the original sequence.

b) $\lim_{n\to\infty}\frac{5n^2-3n+2}{n^2-n} $

Divide the top and the bottom by $n^2$ and evaluate:

$\lim_{n\to\infty}\frac{5-3\frac{1}{n}+2\frac{1}{n^2}}{1-\frac{1}{n}} =
(\lim_{n\to\infty}5-3\lim_{n\to\infty}\frac{1}{n}+2\lim_{n\to\infty}\frac{1}{n^2})
\cdot \lim_{n\to\infty}\frac{1}{1-\frac{1}{n}} =
(5-0+0)\cdot\frac{1}{1-\lim_{n\to\infty}\frac{1}{n}} = 5$. 
The sequence converges to 5.

c) $\lim_{n\to\infty}\frac{n^2+5n+2}{5n^3+n} $

Divide the top and the bottom by $n^3$ and evaluate:

$\lim_{n\to\infty}\frac{\frac{1}{n}+5\frac{1}{n^2}+2\frac{1}{n^3}}{5+\frac{1}{n^2}} =
(\lim_{n\to\infty}\frac{1}{n}+5\lim_{n\to\infty}\frac{1}{n^2}+2\lim_{n\to\infty}
\frac{1}{n^3})\cdot\lim_{n\to\infty}\frac{1}{5+\frac{1}{n^2}} = 
(0+5\cdot0+2\cdot0)\cdot\lim_{n\to\infty}\frac{1}{5+\frac{1}{n^2}} = 
0\cdot\frac{1}{5+\lim_{n\to\infty}\frac{1}{n^2}}=0\cdot\frac{1}{5} = 0$.
The sequence converges to 0.


d) $\lim_{n\to\infty}(-1)^nn $

This sequence diverges. It is the product of an oscillating
value with constant magnitude and a value that grows without bound.

e) $\lim_{n\to\infty}\frac{1}{(-1)^nn+2n} $

We can use Proposition 2.1.13 and part (d) to evaluate:

$\frac{1}{\lim_{n\to\infty}(-1)^nn+\lim_{n\to\infty}2n} =
\frac{1}{0+2\lim_{n\to\infty}n} = \frac{\lim_{n\to\infty}\frac{1}{n}}{2} =
\frac{0}{2} = 0$. The sequence converges to 0.

f) $\lim_{n\to\infty}\sqrt[3]{n+1}-\sqrt[3]{n} $

We rewrite the expression to utilize the $\frac{1}{n^p}$ property:

$\lim_{n\to\infty}(n+1)^\frac{1}{3}-n^\frac{1}{3} =
\lim_{n\to\infty}(n+1)^\frac{1}{3}-\lim_{n\to\infty}n^\frac{1}{3} =
0 - 0 = 0$. The sequence converges to 0.

\subsection*{Problem 11}
Let $a>1$. We will show that $\lim_{n\to\infty}a^n=\infty$.
To satisfy the definition of diverging to $\infty$, we will show
$\forall B > 0, \exists N$ such that $n > N \implies a^n > B$. 
If we choose $N = \log_a(B)$, we can see that 
$n > \log_a(B) \implies a^n > a^{log_a(B)} = B$.

Now consider if $0<a<1$. We will show that in this case, $\lim_{n\to\infty}a^n=0$
by showing that $\forall \varepsilon > 0, \exists N$ such that
$\left| a^n \right| < \varepsilon$. We can choose $N = log_a(\varepsilon)$.
So, when $n > N$, $n > log_a(\varepsilon)$. Now we will raise $a$ to the power 
of both sides, but we have to switch the inequality sign since $a^n$ is monotone
decreasing. So, $a^n = |a^n| < a^{log_a(\varepsilon)} = \varepsilon$, and we
conclude that the sequence converges to 0.

\subsection*{Problem 13}
Let $(a_n)$ and $(b_n)$ be sequences where $\lim_{n\to\infty}a_n=L$ and 
$\lim_{n\to\infty}b_n=\infty$. We will show that if $L < 0$ then
$\lim_{n\to\infty}a_nb_n=-\infty$.

We want to show that $\forall B<0, \exists N$ such that
$n > N \implies a_nb_n < B$. 
By the definitions of convergence to $L$ and convergence to $\infty$, we know
that $\forall \varepsilon>0, \exists N_1$ such that $n<N_1 \implies 
|a_n-L|<\varepsilon$ and that $\forall M>0, \exists N_2$ such that
$n > N_2 \implies b_n > M$. In particular, if we select $\varepsilon < L$, we can
find an $N_1$ that ensures $a_n$ is negative for all $n > N_1$.
In addition, if we select $M = \frac{-B}{L}$, we can find an
$N_2$ to ensure $b_n > \left|\frac{B}{L}\right|$ for all $n > N_2$. 
Now we take $N = \max(N_1, N_2)$ which will ensure both conditions
when $n > N$. Now notice that when $n > N$, $a_nb_n < L\cdot\frac{B}{L} = B$. 
So we have shown that $\lim_{n\to\infty}a_nb_n=-\infty$.

\subsection*{Problem 15}

Suppose $(a_n)$ is a sequence with $a_n > 0$ and 
$\lim_{n\to\infty} \frac{a_{n+1}}{a_n} = L$.
We want to show that if $L < 1$, then $\lim_{n\to\infty}a_n=0$.
By separability, there must exist some $\varepsilon > 0$ so that
$L + \varepsilon < 1$. Further, by the definition of convergence to $L$, 
there exists some $N \in \N$ such that 
$n \geq N \implies \frac{a_{n+1}}{a_n} < L + \varepsilon$.
Specifically, $\frac{a_{N+1}}{a_N}<L+\varepsilon$. We can rearrange this as
$a_{N+1} < (L+\varepsilon)\cdot a_N$. Every ratio between successive terms of $a_n$ past
$n=N$ will be smaller than $L+\varepsilon$, so it is valid to say that, in general,
$a_{N+k} < (L+\varepsilon)\cdot a_{N+k-1}$. If we keep expanding $a_{N+k-1}$ using
the same principle, then we can write $a_{N+k} < (L+\varepsilon)^k\cdot a_N$.
Recall that $a_n > 0$, and note that the expression on the right
side converges to 0 as $k\to\infty$ since $0<(L+\varepsilon)<1$, as shown in Problem 11. 
So, by the Squeeze theorem, we can see that as $k\to\infty$, $a_{N+k}$ converges to 0
as it is always greater than 0 and is less than another sequence that also 
converges to 0.


\subsection*{Problem 16}

In the case where $L > 1$, $a_n\to\infty$ because even if the ratio between
successive terms converges to 1.01, the sequence will grow without bound.
If $L = 1$, the sequence $a_n$ will converge to some value. The ratio between
successive terms approaches 1, so the sequence will tend to a single value as 
$n\to\infty$.


\end{document}
