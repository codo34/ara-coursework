\documentclass[11pt]{article}
\usepackage{amsfonts}
\usepackage{amsmath}

\newcommand{\N}{\mathbb{N}}
\newcommand{\Z}{\mathbb{Z}}
\newcommand{\Q}{\mathbb{Q}}
\newcommand{\R}{\mathbb{R}}
\renewcommand{\P}{{\cal P}}

\pagestyle{empty}


\begin{document}

\setlength{\parindent}{0pt}
\setlength{\parskip}{9pt}


\section*{Math 351: Homework 2 (Due September 21)}
\section*{Name: Jack Ellert-Beck}

\bigskip

\subsection*{Section 0.5}
\subsection*{Problem 7}

We will show by induction that for all integers $n \geq 1, n < 2^n$.

For the base case, let $n=1$. $2^n = 2^1 = 2 > 1$, so the property holds.

Now we will show that if $n < 2^n$ then $n+1 < 2^{n+1}$. We can add one to both sides
of the first inequality: 
$n+1 < 2^n + 1 < 2^n + 2^n < 2^n\cdot 2 = 2^{n+1}$. This completes the inductive step.
Thus, by mathematical induction, the property is true for all $n \geq 1$.

\subsection*{Problem 8}

Let $\{F_n\}$ be the Fibonacci sequence. We will prove that for all $n \in \N$,
\[ \sum^n_{i=0}F_i = F_{n+2}-1. \]

We proceed by induction. Let $n=1$ and recall that $F_1=1$ and $F_3=2$.
We see that $F_1 = 1 = 2 - 1 = F_3-1$.

For the inductive step we assume $ \sum^n_{i=0}F_i = F_{n+2}-1 $, and we want
to show that $ \sum^{n+1}_{i=0}F_i = F_{n+3}-1 $. In addition,
recall that the definition of $\{F_n\}$ includes the statement 
$F_n = F_{n-1} + F_{n+2}$. Consider the following equalities:

\[
\begin{array}{rl}
\sum^{n+1}_{i=0}F_i &= \sum^n_{i=0}F_i + F_{n+1} \\
&= F_{n+2} - 1 + F_{n+1} \\
&= (F_{n+2} + F_{n+1}) - 1 \\
&= F_{n+3} - 1.
\end{array}
\]
By mathematical induction, this property must hold for all $n \in \N$.

\subsection*{Problem 9}
We will show that for all integers $n \geq 1$,
\[ \sum^n_{i=1}\frac{1}{2^i} = 1-\frac{1}{2^n}. \]

We proceed by induction. Consider the base case where $n=1$. We get
$\frac{1}{2^1} = \frac{1}{2} = 1 - \frac{1}{2} = 1 - \frac{1}{2^1}$, so the
base case is true.

For the inductive step we assume $\sum^n_{i=1}\frac{1}{2^i} = 1-\frac{1}{2^n}$
and we want to show that $\sum^{n+1}_{i=1}\frac{1}{2^i} = 1-\frac{1}{2^{n+1}}$.
Note that the following equalities hold:

\[
\def\arraystretch{1.5}
\begin{array}{rl}
\sum^{n+1}_{i=1}\frac{1}{2^i} &= \sum^{n}_{i=1}\frac{1}{2^i} + \frac{1}{2^{n+1}} \\
&= 1 - \frac{1}{2^n} + \frac{1}{2^{n+1}} \\
&= 1 - \frac{2^{n+1} - 2^n}{2^n\cdot 2^{n+1}} \\
&= 1 - \frac{2\cdot 2^n - 2^n}{2^n\cdot 2^{n+1}} \\
&= 1 - \frac{2^n}{2^n\cdot 2^{n+1}} \\
&= 1 - \frac{1}{2^{n+1}}.
\end{array}
\]
So, by mathematical induction, this is true for all $n \geq 1$.

\subsection*{Problem 11}

We will show that for all integers $n \geq 1$,
\[ \sum^n_{i=1}i^2= \frac{n(n+1)(2n+1)}{6}. \]

We can show this by induction. We let $n=1$ for the base case, and we see that
$1^2 = 1 = \frac{6}{6} = \frac{1\cdot 2\cdot 3}{6}$.

For the inductive step we assume $\sum^n_{i=1}i^2= \frac{n(n+1)(2n+1)}{6}$
and we want to show that 
$\sum^{n+1}_{i=1}i^2= \frac{(n+1)(n+2)(2(n+1)+1)}{6}=\frac{(n+1)(n+2)(2n+3)}{6} $.
Consider the following equalities:

\[
\def\arraystretch{1.5}
\begin{array}{rl}
\sum^{n+1}_{i=1}i^2 &= \sum^{n}_{i=1}i^2 + (n+1)^2 \\
&= \frac{n(n+1)(2n+1)}{6} + (n+1)^2 \\
&= \frac{n(n+1)(2n+1) + 6(n+1)^2}{6} \\
&= \frac{(n+1)(n(2n+1) + 6(n+1))}{6} \\
&= \frac{(n+1)(2n^2+7n+6)}{6} \\
&= \frac{(n+1)(n+2)(2n+3)}{6}.
\end{array}
\]
By mathematical induction, this property is true for all $n \geq 1$.

\subsection*{Section 1.3}
\subsection*{Problem 7}
\subsection*{Argue that the set of finite sequences of 0's and 1's is countable}
\subsection*{Argue that the set of infinite sequences of 0's and 1's is not countable}
\subsection*{Problem 11}

\subsection*{Section 2.1}
\subsection*{Problem 1}
\subsection*{Problem 2}
\subsection*{Problem 4}
\subsection*{Problem 6}


\end{document}
