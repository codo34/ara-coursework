\documentclass[11pt]{article}
\usepackage{amsfonts}
\usepackage{amsmath}
\usepackage{amssymb}
\usepackage{enumitem}

\newcommand{\N}{\mathbb{N}}
\newcommand{\Z}{\mathbb{Z}}
\newcommand{\Q}{\mathbb{Q}}
\newcommand{\R}{\mathbb{R}}
\renewcommand{\P}{{\cal P}}

\pagestyle{empty}


\begin{document}

\setlength{\parindent}{0pt}
\setlength{\parskip}{9pt}


\section*{Math 351: Homework 5  Due Friday October 19 }
\subsection*{Jack Ellert-Beck}

\bigskip

\subsection*{Sections 3.2 and 3.3}

Section 3.2 Exercises (pg 112-113) work 6,7,8,9,10

\begin{enumerate}
\setcounter{enumi}{5}

\item Let $A\subsetneq\R$. When is $A^\circ = \overline{A}$?

We know that $A^\circ\subseteq A\subseteq \overline{A}$, so 
$A^\circ = \overline{A} = A$ in this case. We have previously shown that
$A=A^\circ$ if and only if $A$ is open, and that 
$A=\overline{A}$ if and only if $A$ is closed. So, $A$
must be both open and closed. The only sets for which this can be true are
$\R$ and $\varnothing$, and since we have $A\subsetneq \R$, the only case
where this is possible is $A=\varnothing$.

\item Is there a set whose interior is empty and whose closure is $\R$?

The set $\Q$ satisfies this property. Pick any $q\in\Q$ and fix it. For
any $\varepsilon>0$, the open ball $B(q,\varepsilon)$ will contain a
nonrational real number due to the order completeness of $\R$. Thus,
$\Q$ contains no interior points. In addition, we know that $\R$ consists
of all values to which sequences of rational numbers converge, so every
real number is the limit of some sequence in $\Q$. Thus every $x\in\R$
is a closure point of $\Q$. So $\Q$ is a set whose interior is empty and
whose closure is $\R$.  

\item Find the closure and interior of each of the following sets:

\begin{enumerate}
\item{$A=[0,1)\cup(1,2)$}

$\overline{A}=[0,2],\ A^\circ=(0,1)\cup(1,2)$
\item{$\N$}

$\overline{\N}=\N,\ \N^\circ=\varnothing$
\item{$A=\{\frac{1}{n}:n\in\N\}$}

$\overline{A}=A\cup0,\ A^\circ=\varnothing$
\item{$A=\Q\cap(0,1)$}

$\overline{A}=[0,1],\ A^\circ=\varnothing$
\item{$A=\R\setminus\Q$}

$\overline{A}=\R,\ A^\circ=\varnothing$ 

Recall that $\Q$ is dense in $\R$, so $A$ has no interior points.
\item{$A=\{\frac{p}{2^n}:n\in\N,p=0,\ldots,2^n\}$}

$\overline{A}=\R,\ A^\circ=\varnothing$

$A$ is the set of dyadic rationals in $[0,1]$. In Example 2.2.6, we define
a function $d:[0,1]\to\{0,1\}^\N$ that maps every real number in $[0,1]$
to a sequence of dyadic intervals of increasing rank. For any 
$x\in [0,1]$, if we take the sequence
of the left endpoints of these intervals determined by $d(x)$,
this sequence will be a subset
of our set $A$ and will converge to $x$.
\end{enumerate}
\item{Prove or disprove $\overline{(A\cup B)}=\overline{A}\cup\overline{B}$.}

We will show that these sets are equal by showing that they are subsets
of one another. First, let $x\in\overline{(A\cup B)}$. This means that
$x$ is a closure point of $A\cup B$. Thus there exists some sequence 
$a_n\in(A\cup B)$ that converges to $x$. There are two possible cases:
either $a_n$ is a subset of A or B, or $(a_n)$ contains elements from $A$
and elements from $B$. If it is the former case, assume without loss of
generality that $(a_n)\subseteq A$ (since the following argument is also
valid for $B$). Thus, $x$ is also a closure point of $A$, so
$x\in\overline{A}\subseteq(\overline{A}\cup\overline{B})$. In the case
where $(a_n)$ contains elements from both $A$ and $B$, take either the
subsequence $(a_n)\cap A$ or $a_n\cap B$, whichever has an infinite
number of elements. We know that if $a_n\to x$, any subsequence of $a_n$
also converges to $x$. Thus we can find a sequence in either $A$ or $B$
that converges to $x$, so $x$ is a closure point of either $A$ or $B$, and it
follows that $x\in(\overline{A}\cup\overline{B})$. 
Since any $x$ in
$\overline{A\cup B}$ is in $\overline{A}\cup\overline{B}$,
$\overline{A\cup B}\subseteq(\overline{A}\cup\overline{B})$.

Now take $x\in(\overline{A}\cup\overline{B})$. This means $x$ is either
a closure point of $A$ or a closure point of $B$, so there is some sequence
$(a_n)$ of terms in either $A$ or $B$ that converges to $x$. Each $a_n$ would
also be in $A\cup B$, however, which means that $x$ is also a closure point 
of $A\cup B$, so $x\in\overline{A\cup B}$.
Hence $(\overline{A}\cup\overline{B})\subseteq\overline{A\cup B}$.
Since we have both sets as subsets of one another, the sets must be equal.

\item{Prove or disprove $\overline{(A\cap B)}=\overline{A}\cap\overline{B}$.}

We disprove this by providing a counterexample.
Let $A=(0,1)$ and $B=(1,2)$. $\overline{(A\cap B)} = \overline{(0,1)\cap(1,2)}
= \overline{\varnothing} = \varnothing$.
But, $\overline{A}\cap\overline{B} = \overline{(0,1)}\cap\overline{(1,2)}
= [0,1]\cap[1,2] = {1} \neq \varnothing$.

\end{enumerate}

Section 3.3 Exercises (pg 116-117) work 7

\begin{enumerate}
\setcounter{enumi}{6}
\item{Find the closure and interior of each of the following sets as
subsets of $\R^2$.}
\begin{enumerate}
\item{$A=\N\times\Q$}

$\overline{A}=\N\times\R,\ A^\circ=\varnothing$
\item{$A=\Z\times\Z$}

$\overline{A}=A,\ A^\circ=\varnothing$
\item{$A=\Q\times (0,1)$}

$\overline{A}=\R\times [0,1],\ A^\circ=\varnothing$
\item{$A=\{(x, \sin (\frac{1}{x})):0<x<\pi\}$}

$\overline{A}=A\cup(\{0\}\times[-1,1]),\ A^\circ=\varnothing$

For every $y\in [-1,1]$
there is a sequence in $A$ converging to the element $(0,y)$, namely
the set of all $(x,y)$ where $x$ is a solution to the equation
$\sin(\frac{1}{x})-y=0$, enumerated in decreasing order of $x$. Thus
all $(0,y)$ are closure points.
Also, every point in $A$ is a boundary point of $A$. 
\end{enumerate}
\end{enumerate}

\subsection*{Interiors}

For $\R^n$, what is the relationship between $A^\circ$ and $({\bar A})^\circ$? Are they equal? One contained in the other? Prove any claim you make. 

Recall that $A\subseteq\overline{A}$. Thus all interior points of $A$ must
be elements of $\overline{A}$, and any open balls
$B(x,\varepsilon)\subseteq A$ are also subsets of $\overline{A}$. It
follows that all interior points of $A$ are also interior points of 
$\overline{A}$. So, $A^\circ\subseteq (\overline{A})^\circ$. We cannot,
however, establish this relationship in the other direction.
For instance, take $\Q^n$. $(\Q^n)^\circ=\varnothing$ but
$(\overline{\Q^n})^\circ = (\R^n)^\circ = \R^n \neq \varnothing$.

\subsection*{Compactness 1}

Suppose $K$, $F$ are subsets of $\R$, and that $K$ is compact and $F$ is closed. Are the following definitely compact, definitely closed, or possibly neither.

a)\quad $K\cap F$

The intersection of a bounded set with any other set will be bounded,
and any intersection of closed sets will be closed, so
$K\cap F$ will be closed and bounded, and thus compact by Heine-Borel.

b)\quad $\overline{K^c \cup F^c}$   

This set will definitely be closed as it is the closure of some set.
However, since $K$ is bounded, $K^c$ will not be bounded, and neither will
its union with any other set. The closure of this set will be a superset
of an unbounded set, so it will itself be unbounded, and thus not compact.

c)\quad $K\setminus F$

If $K$ and $F$ are disjoint, then $K\setminus F = K$ is 
compact.
However, in the case where $K=[0,1],\ F=[\frac{1}{2},\frac{3}{2}]$,
we get $K\setminus F = [0,\frac{1}{2})$, which is not closed and 
therefore not compact. So $K\setminus F$ can be compact but it need not be.

d)\quad $\overline{K \cap F^c}$

Note that $\overline{K\cap F^c}=\overline{K\setminus F}$.
Since $K\setminus F$ is a subset of $K$, it is definitely bounded.
The set in question is therefore the closure of a bounded set,
so it is compact.


\subsection*{Compactness 2}

Consider the set comprised of closed intervals

$$\textstyle  \ldots  \cup [{1\over 6}, {1\over 5}] \cup [{1\over 4},{1\over 3}] \cup  [{1\over 2},1] $$

That is, let $I_n= [{1\over 2n},{1\over 2n-1}]$ for $n\in \N$ and take $B=\bigcup_{n\in\N} I_n $. 

a) Show $B$ is bounded but not closed. 

Note that, for any $n$, both endpoints of the interval $I_n$ are positive
numbers less than 1. $B\subseteq [0,1]$, so it is bounded. To show that
$B$ is not closed, we show that $B^c$ is not open.
Since $0\notin B,\ 0\in B^c$. Now pick any $\varepsilon>0$ and consider
the open ball $B(0,\varepsilon)$. Recall that $\lim_{n\to\infty}\frac{1}{n}=0$
means that $\forall \varepsilon>0\ \exists N$ such that
$n>N\implies\frac{1}{n}<\varepsilon$. So, it follows that there are an
infinite number of consecutive $\frac{1}{n}$ in $B(0,\varepsilon)$
and thus an infinite number of intervals $I_n$ contained in $B(0,\varepsilon)$.
Hence 0 is not an interior point of $B^c$, so $B^c$ is not open, which means
$B$ is not closed.

b) Find a sequence $\{b_n\}\subset B$ that converges to a point not contained in $B$.

Define $\{b_n\}_{n\in\N}$ such that $b_n = \frac{1}{\frac{(2n)+(2n-1)}{2}}
= \frac{1}{2n-\frac{1}{2}}$. Note that for all $n\in\N$, $b_n\in I_n$, so
${b_n}\subseteq B$. We find the limit of $b_n$:
$\lim_{n\to\infty}b_n=\lim_{n\to\infty}\frac{1}{2n-\frac{1}{2}}=0$, which is
not an element of $B$.

c) Find an open cover of $B$ that cannot be reduced to a finite open cover. 

Define the interval
$I'_n = (\frac{1}{2n}-\frac{1}{100n}, \frac{1}{2n-1}+\frac{1}{100n})$.
Note that for all $n\in\N$, $I'_n$ covers $I_n$ and does not cover any
$I_k$ for $k\neq n$. Let $G=\bigcup_{n\in\N}I'_n$. Since $G$ is a union
of open sets, $G$ is open, so $G$ is an open cover of $B$. Removing any
$I'_n$ from $G$ leaves $I_n$ uncovered, so $G$ cannot be reduced to a 
finite open cover.

d) Show $\tilde B = B\cup\{0\}$ is compact.

We previously showed that $B$ is bounded, so $\tilde B$ is bounded since
it adds a finite number of elements to $B$. We will show that $\tilde B$ is
closed by showing that $\tilde B^c$ is open. We can write $\tilde B^c=
(-\infty,0)\cup(1,\infty)\cup
\bigcup_{n\in\N}\left(\frac{1}{2n+1},\frac{1}{2n}\right)$. This is a union
of open sets and is thus open, so $\tilde B$ is closed. By Heine-Borel,
$\tilde B$ is thus compact.

e) How does the addition of the point $\{0\}$ undo your answers in (b) and (c)? That is, why can't you make similar examples that work for $\tilde B$?

In part (b), 0 was the only value outside of $B$ where we could construct
a sequence contained in $B$ that converges to it, because it was the only
boundary point of $B^c$. In part (c), we can still construct open covers
of $\tilde B$. However, due to the argument from part (a), any open set
containing 0 will also contain an infinite number of $I_n$. So any open
cover of $\tilde B$ can be reduced to the cover that includes whatever
open interval was used to cover 0, plus the open intervals
used to cover the finitely many intervals $I_n$ that haven't already been
covered.


\subsection*{Compactness 3}

Prove that, given a bounded open set $A$ and an $\varepsilon > 0$,
there exists a finite union of closed intervals $\bigcup C_n \subseteq A$
such that $A\setminus\bigcup C_n$ does not contain any balls of radius
$\varepsilon$.

Let $a$ be a lower bound of $A$ and $b$ be an upper bound of $A$. It follows
that $A\subseteq(a-100,b+100)$. So, the set $G=[a-100,b+100]\cap A^c$
is compact, as it is of the form $K\cap F$ from \textbf{Compactness 1}.
We can cover $G$ with $\varepsilon$ balls, and, since $G$ is compact,
reduce this cover to a finite cover $B=\bigcup B_n$. Now consider
$B^c\cap[a-100,b+100]$. Since $B$ is a finite collection of open balls
of fixed size, this set is a finite union of closed intervals,
$\bigcup C_n$. Further, this set is a subset of $A$, since its complement
in $[a-100,b+100]$ covers $A$'s complement in the same interval.
Now, notice that if there is a ball of radius epsilon in
$A\setminus \bigcup C_n$, it would have to be disjoint from $G$ but would
be part of the cover $B$, which means we can still cover $G$ without that ball
and repeat the construction with a modified cover.
Hence we can construct $\bigcup C_n$ such that $A\setminus\bigcup C_n$
contains no balls of radius $\varepsilon$.

\subsection*{Compactness 4}

Work section 3.4 exercise 3 (page 122). 

Let $E$ and $F$ be two disjoint nonempty compact sets in $\R$. We will show
that there exists a $\delta>0$ such that for every interval $I$, if
$|I|<\delta$ then $I\cap E = \varnothing$ or $I\cap F=\varnothing$.
Suppose for contradiction that this is false. Then for all $\delta>0$,
there exists an interval $I$ with $|I|<\delta$ and $I\cap E\neq \varnothing,\
I\cap F\neq \varnothing$. We will construct sequences $\{e_n\}$ and
$\{f_n\}$ by considering points in $I$ for different values of $\delta$.
Pick some $\delta_1$ and find an interval $I_1$ satisfying the conditions.
This interval will contain some $e_1\in E$ and $f_1\in F$. Now pick any
positive $\delta_2<\delta_1$ and find a corresponding $I_2$, containing
some $e_2$ and $f_2$. Continuing this process, we create sequences 
$\{e_n\}$ and $\{f_n\}$ where each $e_n$ and $f_n$ is in $I_n$, and
$|I_n|\to0$. It follows that $0\leq|e_n-f_n|\leq|I_n|$, so $|e_n-f_n|\to0$ by
the Squeeze Property. Since $E$ and $F$ are compact, they are bounded, so
$e_n$ and $f_n$ converge, and in fact converge to the same $L$. But $E$ and 
$F$ are also closed, and $L$ is a closure point of both $E$ and $F$, so
$L$ is in $E$ and $F$, which contradicts our assumption that they are
disjoint sets.

The property is not necessarily true if $E$ and $F$ are closed but not
compact. 

Let $E = \bigcup_{n\in\N}[2n-1, 2n]$ and
$F = \bigcup_{n\in\N}[2n+\frac{1}{3n}, 2n+\frac{1}{2}]$. For both sets,
their complements are unions of open sets, so $E$ and $F$ are both closed.
For any $\delta>0$, find $n$ such that $\frac{1}{3n}<\delta$. Now take
$I=[2n,2n+\frac{1}{3n}]$, which contains elements from both $E$ and $F$. 

\subsection*{Compactness 5}

And now for something completely different. Consider the natural numbers $\N$ with the trivial metric (defined in example 3.1.7 on page 107). 

Show that with this metric, the set $\N$ is closed and bounded, but not sequentially compact.

Define $d(x,y)$ to be the trivial metric, where $d(x,y)=0 \iff x=y$ and
$d(x,y) = 1$ otherwise.
Since $\N$ is the universal set of the metric space in this case, $\N$
is closed. Note that for any $x,y\in\N$, $d(x,y)\leq1$, so $\N$ is bounded.
Now consider the sequence $\{a_n\}$ where $a_n = n$. For any choice of
$\{n_k\}$, the elements of $\{a_{n_k}\}$ fail to get closer to one another
than $\varepsilon$ when $\varepsilon<1$. Thus there is no subsequence of
$\{a_n\}$ which converges, so $\N$ cannot be sequentially compact
in this metric space.

\subsection*{Continuity 1}

From section 4.1 exercises (pg 143), work 4 and 8. See example 4.1.7.

\subsection*{Continuity 2}

From section 4.1 exercises (pg 143), work 5.

\end{document}
