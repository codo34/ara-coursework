\documentclass[11pt]{article}
\usepackage{amsfonts}
\usepackage{amsmath}
\usepackage{amssymb}
\usepackage{enumitem}

\newcommand{\N}{\mathbb{N}}
\newcommand{\Z}{\mathbb{Z}}
\newcommand{\Q}{\mathbb{Q}}
\newcommand{\R}{\mathbb{R}}
\renewcommand{\P}{{\cal P}}

\pagestyle{empty}


\begin{document}

\setlength{\parindent}{0pt}
\setlength{\parskip}{9pt}


\section*{Math 351: Homework 5  Due Friday October 19 }

\bigskip

\subsection*{Sections 3.2 and 3.3}

Section 3.2 Exercises (pg 112-113) work 6,7,8,9,10

\begin{enumerate}
\setcounter{enumi}{5}

\item Let $A\subsetneq\R$. When is $A^\circ = \overline{A}$?

We know that $A^\circ\subseteq A\subseteq \overline{A}$, so 
$A^\circ = \overline{A} = A$ in this case. We have previously shown that
$A=A^\circ$ if and only if $A$ is open, and that 
$A=\overline{A}$ if and only if $A$ is closed. So, $A$
must be both open and closed. The only sets for which this can be true are
$\R$ and $\varnothing$, and since we have $A\subsetneq \R$, the only case
where this is possible is $A=\varnothing$.

\item Is there a set whose interior is empty and whose closure is $\R$?

The set $\Q$ satisfies this property. Pick any $q\in\Q$ and fix it. For
any $\varepsilon>0$, the open ball $B(q,\varepsilon)$ will contain a
nonrational real number due to the order completeness of $\R$. Thus,
$\Q$ contains no interior points. In addition, we know that $\R$ consists
of all values to which sequences of rational numbers converge, so every
real number is the limit of some sequence in $\Q$. Thus every $x\in\R$
is a closure point of $\Q$. So $\Q$ is a set whose interior is empty and
whose closure is $\R$.  

\item Find the closure and interior of each of the following sets:

\begin{enumerate}
\item{$A=[0,1)\cup(1,2)$}

$\overline{A}=[0,2],\ A^\circ=(0,1)\cup(1,2)$
\item{$\N$}

$\overline{\N}=\N,\ \N^\circ=\varnothing$
\item{$A=\{\frac{1}{n}:n\in\N\}$}

$\overline{A}=A\cup0,\ A^\circ=\varnothing$
\item{$A=\Q\cap(0,1)$}

$\overline{A}=[0,1],\ A^\circ=\varnothing$
\item{$A=\R\setminus\Q$}

$\overline{A}=\R,\ A^\circ=\varnothing$ 

Recall that $\Q$ is dense in $\R$, so $A$ has no interior points.
\item{$A=\{\frac{p}{2^n}:n\in\N,p=0,\ldots,2^n\}$}

$\overline{A}=\R,\ A^\circ=\varnothing$

$A$ is the set of dyadic rationals in $[0,1]$. In Example 2.2.6, we define
a function $d:[0,1]\to\{0,1\}^\N$ that maps every real number in $[0,1]$
to a sequence of dyadic intervals of increasing rank. For any 
$x\in [0,1]$, if we take the sequence
of the left endpoints of these intervals determined by $d(x)$,
this sequence will be a subset
of our set $A$ and will converge to $x$.
\end{enumerate}
\item{Prove or disprove $\overline{(A\cup B)}=\overline{A}\cup\overline{B}$.}

We will show that these sets are equal by showing that they are subsets
of one another. First, let $x\in\overline{(A\cup B)}$. This means that
$x$ is a closure point of $A\cup B$. Thus there exists some sequence 
$a_n\in(A\cup B)$ that converges to $x$. There are two possible cases:
either $a_n$ is a subset of A or B, or $(a_n)$ contains elements from $A$
and elements from $B$. If it is the former case, assume without loss of
generality that $(a_n)\subseteq A$ (since the following argument is also
valid for $B$). Thus, $x$ is also a closure point of $A$, so
$x\in\overline{A}\subseteq(\overline{A}\cup\overline{B})$. In the case
where $(a_n)$ contains elements from both $A$ and $B$, take either the
subsequence $(a_n)\cap A$ or $a_n\cap B$, whichever has an infinite
number of elements. We know that if $a_n\to x$, any subsequence of $a_n$
also converges to $x$. Thus we can find a sequence in either $A$ or $B$
that converges to $x$, so $x$ is a closure point of either $A$ or $B$, and it
follows that $x\in(\overline{A}\cup\overline{B})$. 
Since any $x$ in
$\overline{A\cup B}$ is in $\overline{A}\cup\overline{B}$,
$\overline{A\cup B}\subseteq(\overline{A}\cup\overline{B})$.

Now take $x\in(\overline{A}\cup\overline{B})$. This means $x$ is either
a closure point of $A$ or a closure point of $B$, so there is some sequence
$(a_n)$ of terms in either $A$ or $B$ that converges to $x$. Each $a_n$ would
also be in $A\cup B$, however, which means that $x$ is also a closure point 
of $A\cup B$, so $x\in\overline{A\cup B}$.
Hence $(\overline{A}\cup\overline{B})\subseteq\overline{A\cup B}$.
Since we have both sets as subsets of one another, the sets must be equal.

\item{Prove or disprove $\overline{(A\cap B)}=\overline{A}\cap\overline{B}$.}

We disprove this by providing a counterexample.
Let $A=(0,1)$ and $B=(1,2)$. $\overline{(A\cap B)} = \overline{(0,1)\cap(1,2)}
= \overline{\varnothing} = \varnothing$.
But, $\overline{A}\cap\overline{B} = \overline{(0,1)}\cap\overline{(1,2)}
= [0,1]\cap[1,2] = {1} \neq \varnothing$.

\end{enumerate}

Section 3.3 Exercises (pg 116-117) work 7

\subsection*{Interiors}

For $\R^n$, what is the relationship between $A^\circ$ and $({\bar A})^\circ$? Are they equal? One contained in the other? Prove any claim you make. 

\subsection*{Compactness 1}

Suppose $K$, $F$ are subsets of $\R$, and that $K$ is compact and $F$ is closed. Are the following definitely compact, definitely closed, or possibly neither.

a)\quad $K\cap F$

b)\quad $\overline{K^c \cup F^c}$   

c)\quad $K\backslash F$

d)\quad $\overline{K \cap F^c}$


\subsection*{Compactness 2}

Consider the set comprised of closed intervals

$$\textstyle  \ldots  \cup [{1\over 6}, {1\over 5}] \cup [{1\over 4},{1\over 3}] \cup  [{1\over 2},1] $$

That is, let $I_n= [{1\over 2n},{1\over 2n-1}]$ for $n\in \N$ and take $B=\bigcup_{n\in\N} I_n $. 

a) Show $B$ is bounded but not closed. 

b) Find a sequence $\{b_n\}\subset B$ that converges to a point not contained in $B$.

c) Find an open cover of $B$ that cannot be reduced to a finite open cover. 

d) Show $\tilde B = B\cup\{0\}$ is compact.

e) How does the addition of the point $\{0\}$ undo your answers in (b) and (c)? That is, why can't you make similar examples that work for $\tilde B$?


\subsection*{Compactness 3}

Prove that any open cover of any closed set can be reduced to a countable open cover. 

(\sl HINT: Let $B$ be a closed set, and define $B_n= B\cap [n-1,n]$ for $n\in\Z$. Then $B_n$ is compact.\rm) 

Use this to work section 3.4 exercise 2 (page 122). 

\subsection*{Compactness 4}

Work section 3.4 exercise 3 (page 122). 

(\sl HINT: Suppose not. Then $\forall \delta>0$, there is an interval $I$ of length $\delta$ with $I\cap E \not= \emptyset$ and $I\cap E \not= \emptyset$. Use this to construct convergent $e_n$ in $E$ and $f_n$ in $F$, with $|e_n-f_n|\to 0$. )  \rm


\subsection*{Compactness 5}

And now for something completely different. Consider the natural numbers $\N$ with the trivial metric (defined in example 3.1.7 on page 107). 

Show that with this metric, the set $\N$ is closed and bounded, but not sequentially compact.





\end{document}
