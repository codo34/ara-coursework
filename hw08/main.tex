\documentclass[11pt]{article}
\usepackage{amsfonts}
\usepackage{amsmath}
\usepackage{amssymb}
\usepackage{enumitem}

\newcommand{\N}{\mathbb{N}}
\newcommand{\Z}{\mathbb{Z}}
\newcommand{\Q}{\mathbb{Q}}
\newcommand{\R}{\mathbb{R}}
\renewcommand{\P}{{\cal P}}

\pagestyle{empty}


\begin{document}

\setlength{\parindent}{0pt}
\setlength{\parskip}{9pt}


\section*{Math 351: Homework 8  Due Friday November 16}
\subsection*{Jack Ellert-Beck}

\bigskip

\section*{Invitation to Real Analysis}

\subsection*{Section 6.1}

Exercise 7: Let $f:[a,b]\to\R$ be a continuous function. Suppose that
$f(x)\geq0$ for all $x\in[a,b]$ and $\int_a^bf=0$. Show that $f(x)=0$
for all $x\in[a,b]$.

Assume that there is some $c\in[a,b]$ where $f(c)>0$. Now pick any $0<\varepsilon<f(c)$.
By the continuity of $f$, we can find a $\delta$ such that
$|x-c|<\delta\implies|f(x)-f(c)|<\varepsilon$. So, pick such a $\delta$. Notice
that $f(c-\delta)>f(x)-\varepsilon>0$ and $f(c+\delta)>f(x)-\varepsilon>0$.
Now define the function
\[
g(x)=\begin{cases}
        0 & a\leq x<c-\delta \\
        f(c)-\varepsilon & c-\delta\leq x\leq c+\delta \\
        0 & c+\delta<b
    \end{cases}
\]
We can see that $\int_a^bg=2\delta(f(c)-\varepsilon)>0$, and that $f(x)\geq g(x)$ for
all $x\in[a,b]$. So, $\int_a^bf\geq\int_a^bg>0$, a contradiction. Thus we conclude
that $f(c)\ngtr0$ for any $c$, so $f(x)=0$ for all $x\in[a,b]$.
 
\subsection*{Section 6.2}

Exercise 1: Let $f:[a,b]\to\R$ be a continuous function. Show that there exists
$c\in(a,b)$ such that
\[ f(c)=\frac{1}{b-a}\int_a^bf \]

First define the function
\[ g(x) = \int_a^xf(t)dt \]
and note that $g$ is continuous on $[a,b]$ and differentiable on $(a,b)$.
So, by the Mean Value Theorem, there exists a $c\in(a,b)$ such that
$g'(c)=\frac{g(b)-g(a)}{b-a}$. Now, by the Fundamental Theorem of Calculus, we have
$g'(c)=f(c)$, and we can rewrite the right hand side as well, giving the equation
$f(c)=\frac{\int_a^bf-\int_a^af}{b-a}=\frac{1}{b-a}(\int_a^bf-0)=\frac{1}{b-a}\int_a^bf$
for some $c\in(a,b)$.

\subsection*{Section 8.2}

Exercise 3: Let $f_n:A\to\R$, for $A\subseteq\R$, be uniformly continuous. Show that
if $(f_n)$ converges uniformly to the function $f$ then $f$ is uniformly continuous.

We wish to show that $f$ is uniformly continuous. Let $\varepsilon>0$. We know
there exists $N\in\N$ such that $n\geq N\implies
|f_n(x)-f(x)|<\frac{\varepsilon}{3}$ for all $x\in A$. Since $f_n$ is uniformly
continuous, there exists $\delta>0$ so that
$|f_N(y)-f_N(x)|<\frac{\varepsilon}{3}$ whenever $|x-y|<\delta$ for any choice of
$x,y\in A$. Then
\[
\def\arraystretch{1.5}
\begin{array}{rl}
|f(y)-f(x)| &= |f(y)-f_N(y)+f_N(y)-f_N(x)+f_N(x)-f(x)| \\
&\leq |f(y)-f_N(y)|+|f_N(y)-f_N(x)|+|f_N(x)-f(x)| \\
&\leq 3\cdot\frac{\varepsilon}{3}=\varepsilon
\end{array}
\]
whenever $|x-y|<\delta$. Therefore $f$ is uniformly continuous on $A$.

Exercise 5: Let $x\in[0,\infty)$ and define
\[
f_n(x)=\frac{nx}{nx+1}
\]
Find the pointwise limit $f(x)=\lim_{n\to\infty} f_n(x)$ for $x\in[0,\infty)$ and prove it.
prove that the convergence is uniform on $[1,\infty)$ and is not uniform
on $[0,\infty)$. Is it uniform on $(0,\infty)$?

The pointwise limit is $f(x)=1$. To prove it, fix $x\in[0,\infty)$. Now take any
$N>\frac{1}{x\varepsilon}$. Now, if $n>N$, then 
\[
\def\arraystretch{1.8}
\begin{array}{rl}
n>\frac{\frac{1}{\varepsilon}-1}{x} &\implies nx> \frac{1}{\varepsilon}-1 \\
&\implies nx+1>\frac{1}{\varepsilon} \\
&\implies \frac{1}{nx+1} < \varepsilon \\
&\implies \left| \frac{nx-nx-1}{nx+1}\right|<\varepsilon \\
&\implies \left| \frac{nx}{nx+1}-1 \right|<\varepsilon
\end{array}
\]
And thus $f_n$ converges pointwise to $f(x)=1$.

Exercise 10: Let $f_n(x)=\frac{\sin(nx)}{\sqrt{n}}$ for $x\in[0,\pi/2]$ and $n\in\N$.
Prove that $f_n$ converges uniformly to 0 on $[0,\pi/2]$, but $f_n'$ does not
converge on $[0,\pi/2]$.

Pick any $\varepsilon>0$ and fix it. Now take $N>\frac{1}{\varepsilon^2}$.
Now we can see that
\[
\def\arraystretch{1.5}
\begin{array}{rl}
\frac{1}{\varepsilon^2} &\implies \frac{1}{\varepsilon}\sqrt{n} \\
&\implies 1<\sqrt{n}\varepsilon \\
&\implies |\sin(nx)|<\sqrt{n}\varepsilon \\
&\implies \left| \frac{\sin(nx)}{\sqrt{n}}\right| < \varepsilon
\end{array}
\]
for any choice of $x\in[0,\pi/2]$. So, $f_n\to f(x)=0$ uniformly.

Now, we can see that $f'_n(x) = \sqrt{n}\cos(nx)$. We will show that
$f'$ fails to converge at $x=0$. Note that, for any $n$,
$f'_n(0)=\sqrt{n\cos(0)}=\sqrt{n}$. As $n\to\infty$, $\sqrt{n}\to\infty$ as well.
So the value of $f'_n(0)\to\infty$ as $n\to\infty$, so $f'_n$ fails to converge
at $x=0$. 

\section*{Applied Topics}


\end{document}
