\documentclass[11pt]{article}
\usepackage{amsfonts}
\usepackage{amsmath}
\usepackage{amssymb}
\usepackage{enumitem}

\newcommand{\N}{\mathbb{N}}
\newcommand{\Z}{\mathbb{Z}}
\newcommand{\Q}{\mathbb{Q}}
\newcommand{\R}{\mathbb{R}}
\renewcommand{\P}{{\cal P}}
\renewcommand{\S}{{\cal S}}

\pagestyle{empty}


\begin{document}

\setlength{\parindent}{0pt}
\setlength{\parskip}{9pt}


\section*{Math 351: Homework 8  Due Friday November 16}
\subsection*{Jack Ellert-Beck}

\bigskip

\section*{Invitation to Real Analysis}

\subsection*{Section 6.1}

Exercise 7: Let $f:[a,b]\to\R$ be a continuous function. Suppose that
$f(x)\geq0$ for all $x\in[a,b]$ and $\int_a^bf=0$. Show that $f(x)=0$
for all $x\in[a,b]$.

Assume that there is some $c\in[a,b]$ where $f(c)>0$. Now pick any $0<\varepsilon<f(c)$.
By the continuity of $f$, we can find a $\delta$ such that
$|x-c|<\delta\implies|f(x)-f(c)|<\varepsilon$. So, pick such a $\delta$. Notice
that $f(c-\delta)>f(x)-\varepsilon>0$ and $f(c+\delta)>f(x)-\varepsilon>0$.
Now define the function
\[
g(x)=\begin{cases}
        0 & a\leq x<c-\delta \\
        f(c)-\varepsilon & c-\delta\leq x\leq c+\delta \\
        0 & c+\delta<b
    \end{cases}
\]
We can see that $\int_a^bg=2\delta(f(c)-\varepsilon)>0$, and that $f(x)\geq g(x)$ for
all $x\in[a,b]$. So, $\int_a^bf\geq\int_a^bg>0$, a contradiction. Thus we conclude
that $f(c)\ngtr0$ for any $c$, so $f(x)=0$ for all $x\in[a,b]$.
 
\subsection*{Section 6.2}

Exercise 1: Let $f:[a,b]\to\R$ be a continuous function. Show that there exists
$c\in(a,b)$ such that
\[ f(c)=\frac{1}{b-a}\int_a^bf \]

First define the function
\[ g(x) = \int_a^xf(t)dt \]
and note that $g$ is continuous on $[a,b]$ and differentiable on $(a,b)$.
So, by the Mean Value Theorem, there exists a $c\in(a,b)$ such that
$g'(c)=\frac{g(b)-g(a)}{b-a}$. Now, by the Fundamental Theorem of Calculus, we have
$g'(c)=f(c)$, and we can rewrite the right hand side as well, giving the equation
$f(c)=\frac{\int_a^bf-\int_a^af}{b-a}=\frac{1}{b-a}(\int_a^bf-0)=\frac{1}{b-a}\int_a^bf$
for some $c\in(a,b)$.

\subsection*{Section 8.2}

Exercise 3: Let $f_n:A\to\R$, for $A\subseteq\R$, be uniformly continuous. Show that
if $(f_n)$ converges uniformly to the function $f$ then $f$ is uniformly continuous.

We wish to show that $f$ is uniformly continuous. Let $\varepsilon>0$. We know
there exists $N\in\N$ such that $n\geq N\implies
|f_n(x)-f(x)|<\frac{\varepsilon}{3}$ for all $x\in A$. Since $f_n$ is uniformly
continuous, there exists $\delta>0$ so that
$|f_N(y)-f_N(x)|<\frac{\varepsilon}{3}$ whenever $|x-y|<\delta$ for any choice of
$x,y\in A$. Then
\[
\def\arraystretch{1.5}
\begin{array}{rl}
|f(y)-f(x)| &= |f(y)-f_N(y)+f_N(y)-f_N(x)+f_N(x)-f(x)| \\
&\leq |f(y)-f_N(y)|+|f_N(y)-f_N(x)|+|f_N(x)-f(x)| \\
&\leq 3\cdot\frac{\varepsilon}{3}=\varepsilon
\end{array}
\]
whenever $|x-y|<\delta$. Therefore $f$ is uniformly continuous on $A$.

Exercise 5: Let $x\in[0,\infty)$ and define
\[
f_n(x)=\frac{nx}{nx+1}
\]
Find the pointwise limit $f(x)=\lim_{n\to\infty} f_n(x)$ for $x\in[0,\infty)$ and prove it.
prove that the convergence is uniform on $[1,\infty)$ and is not uniform
on $[0,\infty)$. Is it uniform on $(0,\infty)$?

The pointwise limit is $f(x)=1$. To prove it, fix $x\in[0,\infty)$. Now take any
$N>\frac{1}{x\varepsilon}$. Now, if $n>N$, then 
\[
\def\arraystretch{1.8}
\begin{array}{rl}
n>\frac{\frac{1}{\varepsilon}-1}{x} &\implies nx> \frac{1}{\varepsilon}-1 \\
&\implies nx+1>\frac{1}{\varepsilon} \\
&\implies \frac{1}{nx+1} < \varepsilon \\
&\implies \left| \frac{nx-nx-1}{nx+1}\right|<\varepsilon \\
&\implies \left| \frac{nx}{nx+1}-1 \right|<\varepsilon
\end{array}
\]
And thus $f_n$ converges pointwise to $f(x)=1$.

Exercise 10: Let $f_n(x)=\frac{\sin(nx)}{\sqrt{n}}$ for $x\in[0,\pi/2]$ and $n\in\N$.
Prove that $f_n$ converges uniformly to 0 on $[0,\pi/2]$, but $f_n'$ does not
converge on $[0,\pi/2]$.

Pick any $\varepsilon>0$ and fix it. Now take $N>\frac{1}{\varepsilon^2}$.
Now we can see that
\[
\def\arraystretch{1.5}
\begin{array}{rl}
\frac{1}{\varepsilon^2} &\implies \frac{1}{\varepsilon}\sqrt{n} \\
&\implies 1<\sqrt{n}\varepsilon \\
&\implies |\sin(nx)|<\sqrt{n}\varepsilon \\
&\implies \left| \frac{\sin(nx)}{\sqrt{n}}\right| < \varepsilon
\end{array}
\]
for any choice of $x\in[0,\pi/2]$. So, $f_n\to f(x)=0$ uniformly.

Now, we can see that $f'_n(x) = \sqrt{n}\cos(nx)$. We will show that
$f'$ fails to converge at $x=0$. Note that, for any $n$,
$f'_n(0)=\sqrt{n\cos(0)}=\sqrt{n}$. As $n\to\infty$, $\sqrt{n}\to\infty$ as well.
So the value of $f'_n(0)\to\infty$ as $n\to\infty$, so $f'_n$ fails to converge
at $x=0$. 

\section*{Applied Topics}

\subsection*{Metric and Norms}

We will prove that if a metric is translation invariant and scales, then it induces a norm
by $\|\phi\|=d(\phi,0)$.
First, we verify the triangle inequality, using the triangle inequality for the metric $d$:
$\|\phi+\mu\|=d(\phi+\mu,0)\leq d(\phi+\mu,\mu)+d(\mu,0)=d(\phi,0)+d(\mu,0)=\|\phi\|+\|\mu\|$.
Next, we verify that the induced norm scales:
$r\|\phi\|=rd(\phi,0)=d(r\phi,r\cdot0)=\|r\phi\|$.
Finally, we see that
$\phi\equiv0\iff d(\phi,0)=0\iff\|\phi\|=0$.

\subsection*{$L^2$ norm}

Let $\S = \{\phi\in C_{[0,\pi]}^0,\phi(0)=\phi(\pi)=0\}$ and define
\[
\|f\|=\sqrt{\int_0^\pi(f(x))^2dx}
\]

We will show that this is a norm. Notice that $(f(x))^2$ is always positive, so
$\|f\|$ is the square root of a non-negative number, so $\|f\|\geq0$ for all $f\in\S$.
In addition, by Exercise 7 above,
$\|f\|=0\implies\int_0^\pi(f(x))^2dx=0\implies f(x)^2=f(x)\equiv0$.
Going the other direction, we can see that if $f(x)\equiv0$, then $\|f\|=0$.
Now consider scalar multiplication. For any $r\geq0$,
$\|rf\|=\sqrt{\int_0^\pi(rf(x))^2dx}=\sqrt{r^2\int_0^\pi(f(x))^2dx}=
r\sqrt{\int_0^\pi(f(x))^2dx}=r\|f\|$.
Now, we prove the triangle inequality. In order to do this, we will show that the
operation $f\cdot g=\int_0^\pi f(x)g(x)dx$ satisfies the properties of a dot product.
We can see that $\|f\|^2=\int_0^\pi f(x)f(x)dx=f\cdot f$. We also have
$f\cdot g=\int_0^\pi f(x)g(x)dx=\int_0^\pi g(x)f(x)dx=g\cdot f$ and
$f\cdot (g+h)=\int_0^\pi f(x)(g(x)+h(x))dx=\int_0^\pi f(x)g(x)+f(x)h(x)dx=
\int_0^\pi f(x)g(x)dx+\int_0^\pi f(x)h(x)dx=f\cdot g+f\cdot h$. So, this operation
has the same properties as the dot product. We now want to show that 
$\|f+g\|\leq\|f\|+\|g\|$. From here we follow the argument Silva uses to show a similar
result in Example 3.1.5. Squaring the left hand side, we get
$\|f+g\|^2=(f+g)\cdot(f+g)=\|f\|^2+2f\cdot g+\|g\|^2$. On the right hand side,
we have
$(\|f\|+\|g\|)^2=\|f\|^2+2\|f\|\|g\|+\|g\|^2$. The proof of the triangle inequality
follows from showing that $f\cdot g\leq\|a\|\|b\|$. Silva proves this case of
the Cauchy-Scwartz Inequality in the example, and we can conclude that
the triangle inequality holds, and thus this is a norm.

\subsection*{Continuity}

Let $\S = \{\phi\in C_{[0,\pi]}^0,\phi(0)=\phi(\pi)=0\}$ and define
$\|\phi\|=\int_0^\pi|\phi(x)|dx$. Consider the functional
$G:\S\to\R$ as $G(\phi)=\int_0^\pi\phi(x)dx$.

Under the metric induced by this norm, $G$ is uniformly continuous.
Pick any $\varepsilon>0$ and fix it. Now choose any positive $\delta<\varepsilon$.
Now, for any choice of $\phi,\mu\in\S$ such that $d(\phi,\mu)<\delta$,
\[
\def\arraystretch{1.5}
\begin{array}{rl}
\varepsilon>\delta &> \int_0^\pi|\phi(x)-\mu(x)|dx \\
&\geq |\int_0^\pi \phi(x)-\mu(x)dx| \\
&= |\int_0^\pi \phi(x)dx-\int_0^\pi\mu(x)dx| \\
&= |G(\phi)-G(\mu)|.
\end{array}
\]
This satisfies the definition for uniform continuity.

\subsection*{Another Norm}


Let $\S = \{\phi\in C_{[0,\pi]}^1,\phi(0)=\phi(\pi)=0\}$ and define
\[
\|\phi\|=\max_{x\in[0,\pi]}|\phi'(x)|
\]

We will show that $\S$ is complete under this norm.
Let $(\phi_n)$ be a Cauchy sequence in $\S$ converging to $\phi$. 
Notice that if $\|\phi\|<\varepsilon$, the derivative of $\phi$ is bounded.
By the Mean Value Theorem, the maximum value of $\frac{\phi(a)-\phi(b)}{a-b}$ is 
$\varepsilon$ too. The least upper bound we can give for $|\phi|$ is thus $\varepsilon\pi$.
By Corollary 8.2.16, $f$ is differentiable, so $f$ is in our set, and we conclude
that the space is complete.


\subsection*{More Continuity}

Consider $\S = \{ \phi:\phi\in C_{[0,\pi]}^0,\phi(0)=0 \}$ with the norm
$\|\phi\|=\int_0^\pi|\phi(x)|dx$.
Consider the operator $G:\S\to\S$ as $G(\phi)(x)=\int_0^x\phi(t)dt$.

We can see that $\phi\in\S\implies G(\phi)\in\S$ because every $G(\phi)(x)$
is differentiable, and thus continuous. In addition, for all $\phi\in\S$,
$G(\phi)(0)=\int_0^0\phi(x)dx=0$. So, $G$ is an operator.

(I think $G$ is continuous but not uniformly contiuous? Don't know how to prove.)

This operator is linear. If we have two functions $\phi,\mu\in\S$, then
$G(\phi+\mu)=\int_0^x\phi(t)+\mu(t) dt=\int_0^x\phi(t)dt+\int_0^x\mu(t)dt=G(\phi)+G(\mu)$.
Further, for a non-negative scalar $r$,
$G(r\phi)=\int_0^xr\phi(t)dt=r\int_0^x\phi(x)dx=rG(\phi)$.
It also has a fixed point: if $f(x)=0$ for all $x$, then $G(f)(x)=0$ for all $x$ as well,
and thus $G(f)=f$.

\subsection*{Completeness}

Show that $C_{[0,\pi]}^0$ is complete under the sup metric. 

We show this by proving that every Cauchy sequence converges to a point in the space.
Recall that the sup metric compares functions $f,g$ by the maximum value of
$f-g$ over the interval. So, if we have a Cauchy sequence $(\phi_n)$, we know that
$\forall\varepsilon>0\ \exists N>0$ such that
$n,m>N\implies\max_{x\in[0,\pi]}|\phi_n(x)-\phi_m(x)|<\varepsilon$.
At any individual $x$,
$|\phi_n(x)-\phi_m(x)|<\max_{x\in[0,\pi]}|\phi_n(x)-\phi_m(x)|<\varepsilon$, so
for that choice of $x$, $\phi_n(x)$ is a Cauchy sequence in $\R$, which means
it converges to some value $\phi(x)$. So $\phi_n\to\phi$ pointwise, and we will now prove
that this $\phi$ is in the set.

Pick an $\varepsilon>0$ and fix it. Since $\phi_n$ is Cauchy, we can find
$N$ such that $n,m>N\implies|\phi_n(x)-\phi_m(x)|<\varepsilon/3$ for all
$x\in[0,\pi]$. Now fix $n=N+1$.
Because $\phi_n$ is continuous on a closed interval, it is uniformly continuous on that
interval. So, for the same $\varepsilon$, there exists a $\delta>0$ such that, for any
$x,y\in[0,\pi]$,
$|x-y|<\delta\implies|\phi_n(x)-\phi_n(y)|<\varepsilon$.
Because of the pointwise convergence from earlier, we can find for any $x$
an $M_x$ such that $m>M_x\implies|\phi_m(x)-\phi(x)|<\varepsilon/3$, and similarly we
can find for any $y$ an $M_y$ so that $m>M_y\implies|\phi_m(y)-\phi(y)|<\varepsilon/3$.
If we choose $m>\max(M_x,M_y)$ and keep $|x-y|<\delta$ we see that
$|\phi_m(x)-\phi_m(y)|+|\phi_m(x)-\phi(x)|+|\phi_m(y)-\phi(y)|=
|\phi(x)-\phi_m(x)+\phi_m(x)+\phi_m(y)-\phi_m(y)-\phi(y)|=
|\phi(x)-\phi(y)|<3\cdot\varepsilon/3=\varepsilon$. This satisfies the definition
of uniform continuity, so $\phi$ is continuous on $[0,\pi]$, which means
it is a point in our set. Thus all Cauchy sequences converge in this space,
so the space is complete.


\end{document}
