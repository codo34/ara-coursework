\documentclass[11pt]{article}
\usepackage{amsfonts}
\usepackage{amsmath}
\usepackage{amssymb}
\usepackage{enumitem}

\newcommand{\N}{\mathbb{N}}
\newcommand{\Z}{\mathbb{Z}}
\newcommand{\Q}{\mathbb{Q}}
\newcommand{\R}{\mathbb{R}}
\renewcommand{\P}{{\cal P}}
\renewcommand{\S}{{\cal S}}

\pagestyle{empty}


\begin{document}

\setlength{\parindent}{0pt}
\setlength{\parskip}{9pt}


\section*{Math 351: Homework 10 Due Friday December 7}
\subsection*{Jack Ellert-Beck}

\bigskip

\subsection*{Chopping}

Use Lemma 3.3.4 to complete the proof of the chopping lemma, Lemma 3.2.1

Lemma 3.3.4: If $f$ is continuous on $[-\pi,\pi]$ then
$\int_{-\pi}^\pi\sin(Nu)f(u)du$ converges uniformly to 0.

We will use this lemma to prove that if $f$ is continuous on $[-\pi,\pi]$
then
$I(x)=\int_{-\pi}^{\pi}\sin(K(u-x))f(u)du$
converges uniformly to 0 as $K\to\infty$.
Using algebra and angle difference formulas, we can rewrite $I(x)$ as
\[
\def\arraystretch{1.5}
\begin{array}{rl}
I(x)=\int_{-\pi}^\pi\sin(Ku-Kx)f(u)du &=
    \int_{-\pi}^\pi\sin(Ku)\cos(Kx)-\cos(Ku)\sin(Kx) \\
&= \cos(Kx)\int_{-\pi}^\pi\sin(Ku)-\sin(Kx)\int_{-\pi}^\pi\cos(Ku).
\end{array}
\]
Note that the period of $\sin(Ku)$ is $\frac{2\pi}{K}$.
In the interval $[-\pi,\pi]$, the function undergoes
$\frac{2\pi}{K}\lfloor K\rfloor$ cycles, and the remaining portion of 
a cycle has total width $2\pi-\frac{2\pi}{K}\lfloor K\rfloor$ which
goes to 0 as $k\to\infty$. So, if we consider the interval that contains
only full cycles, we will be dealing with an integral number of cycles
no matter what, and can basically still apply Lemma 3.3.4, since this
interval will approach $[-\pi,\pi]$ as $k\to\infty$.

Also, we can argue that Lemma 3.3.4 still holds if we replace the $\sin$
with $\cos$. We can define functions $\cos^{+}$ and $\cos^{-}$ as
we did with $\sin$, and the properties of Lemma 3.3.1 will hold for these
analogous functions.
Because we are working on $[-\pi,\pi]$, $\cos$ undergoes a full
cycle on the interval just as $\sin$ does, which helps us prove the analogues
to Lemmas 3.3.2 and 3.3.3. In addition, $\cos$ has the same continuity
properties as $\sin$, so $\int_{-\pi}^\pi\cos(Nu)f(u)du$ converges
uniformly to 0 as $N\to\infty$.

By these two (appropriately hand-wavy) results, we can see that both integrals
in the rewritten form of $I(x)$ above will converge uniformly to 0
as $k\to\infty$, so the whole of $I(x)$ converges uniformly to 0 as
$k\to\infty$, which proves the Chopping Lemma.

\subsection*{0.1 Jensen's}

Exercise 5.1: Prove lemma 5.2.2.

Lemma: If $\rho$ is convex on $I$, and $x_1,x_2\in I$, and equality holds
\[
    \rho(wx_1+(1-w)x_2) = w\rho(x_1)+(1-w)\rho(x_2)
\]
for some $0<w<1$, then either $x_1=x_2$ or $\rho$ is linear on $[x_1,x_2]$.

Pick any $x_1<x_2\in I$.
Pick some $w<1$ and let $c=wx_1+(1-w)x_2$. Note that, by the above equation,
$(c,\rho(c))$ is a point on the secant line through
$(x_1,\rho(x_1))$ and $(x_2,\rho(x_2))$.
Assume for contradiction that $\rho$ is convex
but not linear on $[x_1,c]$ and $[c,x_2]$. Pick
$a\in[x_1,c]$ and $b\in[c,x_2]$ such that $\rho(a)$ and $\rho(b)$ lie
below the secant line through $\rho(x_1)$ and $\rho(x_2)$. 
Now notice that we can write $c$ as $va+(1-v)b$ for some
$v<1$. However, $\rho(c)>v\rho(a)+(1-v)\rho(b)$, which implies
that $\rho$ is not convex on $I$, a contradiction. Thus, if $x_1\neq x_2$,
then $\rho$ must be linear. So, if the equality condition holds,
either $x_1=x_2$ or $\rho$ is linear on $I$. 

Exercise 5.2: Prove lemma 5.2.3.

Lemma: If $\rho$ is a convex function on an interval $I$,
$x_1,\ldots,x_n\in I$,
and $w_1,\ldots,w_n$ are non-negative real numbers with $\sum w_i=1$, then
\[
    \rho\left(\sum w_ix_i\right)\sum w_i\rho(x_i)
\]
if and only if $\rho$ is linear on an interval containing the
$x_i$'s, or all $x_i$'s are the same.

We proceed by induction. The base case is where $n=2$, which is true by
Lemma 5.2.2.

For the inductive step, assume that 
\[
    \rho\left(\sum_{i=0}^n w_ix_i\right)=\sum_{i=0}^n w_i\rho(x_i)
\]
implies that either all $x_i$'s are the same or $\rho$ is linear 
for some $n\geq2$. We will show that
\[
    \rho\left(\sum_{i=0}^{n+1} w_ix_i\right)=\sum_{i=0}^{n+1} w_i\rho(x_i)
\]
implies that either all $x_i$'s are the same or $\rho$ is linear.
We rewrite the latter equation:
\[
\def\arraystretch{1.5}
\begin{array}{rl}
& \rho\left(\sum_{i=0}^{n+1}w_ix_i\right) =
    \rho\left(\left(\sum_{i=0}^nw_ix_i\right)+w_{n+1}x_{n+1}\right) \\
\textrm{by Lemma 5.2.2}&\implies \sum_{i=0}^nw_i\rho(x_i)+w_{n+1}\rho(x_{n+1}) \\
&\implies \sum_{i=0}^{n+1}w_ix_i \\
&\implies \textrm{ and that either $x_i$'s are the same or $\rho$ is linear.}
\end{array}
\]

Exercise 5.3: Prove that convex functions are continuous.

We will show that if a function $\rho:I\to\R$ (where $I=(a,b)$) has the property that, for
$x_1,x_2\in I$ and all $0\leq w\leq1$,
$\rho(wx_1+(1-w)x_2)\leq w\rho(x_1)+(1-w)\rho(x_2)$,
then it is continuous on $(a,b)$, meaning that, at any $c\in I$,
$\forall \varepsilon>0\ \exists\delta$ such that
$|x-a|<\delta\implies|\rho(x)-\rho(c)|<\varepsilon$.

First, we will see that $\rho(x)$ is bounded on $I$.
Let $A=\min(\rho(a),\rho(b))$.
Note that any $c\in[a,b]$ can be written as
$wa+(1-w)b$ when $w=\frac{b-c}{b-a}$. Thus, by convexity,
$\rho(c)\leq w\rho(a)+(1-w)\rho(b)\leq wA+(1-w)=A$.

Pick any $\varepsilon>0$. If
$\varepsilon\geq\max(|\rho(c)-\rho(b)|,|\rho(a)-\rho(c)|)$, then any
choice of $\delta$ satisfies the definition of continuity. So, we now
consider only cases where
$\varepsilon<A-\rho(c)$. Let $c_1=\frac{a-c}{2}$.
If $|\rho(c_1)-\rho(c)|<\varepsilon$, then pick $\delta<|c_1-c|$. If not,
let $c_2=\frac{c_1-c}{c}$. Note that $c<c_2<c_1$.
Keep picking $c_n$ in this way until, for some $N$,
$|\rho(c_N)-\rho(c)|<\varepsilon$. Choosing $0<\delta<c_N$ will make
$|f(c)-f(x)|<\varepsilon$. Thus, $\rho$ is continuous on $I$.

\subsection*{0.2 Euler-Lagrange}

Exercise 4.1: Use Euler-Lagrange to find the extremals for:

a) $\int_1^2y'^2/x^3dx$ with $y(1)=2,y(2)=17$
\[
\def\arraystretch{1.5}
\begin{array}{rl}
    g(x,y,y')=y'^2/x^3 &\textrm{and}\ 
    \frac{\partial g}{\partial y}-\frac{d}{dx}\left(\frac{\partial g}{\partial y'}\right)=0 \\
    &\implies 0-\frac{d}{dx}\left(\frac{2y'}{x^3}\right)=0 \\
    &\implies \frac{2y'}{x^3}=C_0\textrm{ for some constant $C_0$} \\
    &\implies y'=\frac{C_0}{2}x^3 \\
    &\implies y=\int\frac{C_0}{2}x^3dx=C_1x^4+C_2 \\
    \textrm{so }1C_1+C_2=2\textrm{ and }16C_1+C_2=17 &\implies
    C_1=1,\ C_2=1 \\
    &\implies y=x^4+1
\end{array}
\]

b) $\int_0^{\pi/2}y^2-y'^2-2y\sin(x) dx$ with $y(0)=1,y(\pi/2)=1$
\[
\def\arraystretch{1.5}
\begin{array}{rl}
    g(x,y,y')=y^2-y'^2-2y\sin(x) &\textrm{and}\ 
    \frac{\partial g}{\partial y}-\frac{d}{dx}\left(\frac{\partial g}{\partial y'}\right)=0 \\
    &\implies (2y-2\sin(x))-\frac{d}{dx}\left(-2y'\right)=0 \\
    &\implies y''=\sin(x)-y \\
    &\implies y=C_2\sin(x)+C_1\cos(x)-\frac{1}{2}x\cos(x) \\
    &\textrm{ (from Mathematica) } \\
    \textrm{so }0C_2+1C_1-0=1\textrm{ and }1C_2+0C_1-0=1 &\implies
    C_1=1,\ C_2=1 \\
    &\implies y=\sin(x)+\cos(x)-\frac{1}{2}x\cos(x)
\end{array}
\]

c) $\int_0^\pi y'^2+2y\sin(x) dx$ with $y(0)=0,y(\pi)=0$
\[
\def\arraystretch{1.5}
\begin{array}{rl}
    g(x,y,y')=y'^2+2y\sin(x) &\textrm{and}\ 
    \frac{\partial g}{\partial y}-\frac{d}{dx}\left(\frac{\partial g}{\partial y'}\right)=0 \\
    &\implies 2\sin(x)-\frac{d}{dx}\left(2y'\right)=0 \\
    &\implies y''=\sin(x) \\
    &\implies y=-\sin(x)+C_1x+C_2\\
    &\textrm{ (from Mathematica) } \\
    \textrm{so }0C_1+C_2=0\textrm{ and }\pi C_1+C_2=0 &\implies
    C_1=0,\ C_2=0 \\
    &\implies y=-\sin(x)
\end{array}
\]



\end{document}
