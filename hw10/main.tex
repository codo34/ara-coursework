\documentclass[11pt]{article}
\usepackage{amsfonts}
\usepackage{amsmath}
\usepackage{amssymb}
\usepackage{enumitem}

\newcommand{\N}{\mathbb{N}}
\newcommand{\Z}{\mathbb{Z}}
\newcommand{\Q}{\mathbb{Q}}
\newcommand{\R}{\mathbb{R}}
\renewcommand{\P}{{\cal P}}
\renewcommand{\S}{{\cal S}}

\pagestyle{empty}


\begin{document}

\setlength{\parindent}{0pt}
\setlength{\parskip}{9pt}


\section*{Math 351: Homework 10  Due Friday December 7}
\subsection*{Jack Ellert-Beck}

\bigskip

\subsection*{Chopping}

Use Lemma 3.3.4 to complete the proof of the chopping lemma, Lemma 3.2.1

\subsection*{0.1 Jensen's}

Exercise 5.1: Prove lemma 5.2.2.

Lemma: If $\rho$ is convex on $I$, and $x_1,x_2\in I$, and equality holds
\[
    \rho(wx_1+(1-w)x_2)\leq w\rho(x_1)+(1-w)\rho(x_2)
\]
for some $0<w<1$, then either $x_1=x_2$ or $\rho$ is linear on $[x_1,x_2]$.

[BY CONTRADICTION? CONFUSING]

Exercise 5.2: Prove lemma 5.2.3.

Lemma: If $\rho$ is a convex function on an interval $I$,
$x_1,\ldots,x_n\in I$,
and $w_1,\ldots,w_n$ are non-negative real numbers with $\sum w_i=1$, then
\[
    \rho(\sum w_ix_i)\leq\sumw_i\rho(x_i)
\]
with equality holding iff $\rho$ is linear on an interval containing the
$x_i$'s, or all $x_i$'s are the same.

We proceed by induction. The base case is where $n=2$, which is true by
Lemma 5.2.2.

[FINISH THIS ONE BOY]

Exercise 5.3: Prove that convex functions are continuous.

We will show that if a function $\rho:I\to\R$ has the property that, for
$x_1,x_2\in I$ and all $0\leq w\leq1$,
$\rho(wx_1+(1-w)x_2)=w\rho(x_1)+(1-w)\rho(x_2)$,
then it is also continuous, meaning that, at any $a\in I$,
$\forall \varepsilon>0\ \exists\delta$ such that
$|x-a|<\delta\implies|\rho(x)-\rho(a)|<\varepsilon$.

\end{document}
