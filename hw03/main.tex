\documentclass[11pt]{article}
\usepackage{amsfonts}
\usepackage{amsmath}

\newcommand{\N}{\mathbb{N}}
\newcommand{\Z}{\mathbb{Z}}
\newcommand{\Q}{\mathbb{Q}}
\newcommand{\R}{\mathbb{R}}
\renewcommand{\P}{{\cal P}}

\pagestyle{empty}


\begin{document}

\setlength{\parindent}{0pt}
\setlength{\parskip}{9pt}


\section*{Math 351: Homework 3 (Due September 28)}
\section*{Name: Jack Ellert-Beck}

\bigskip

\subsection*{Problem 1}
Let $x_1 = 2$ and define $x_{n+1} = \frac{1}{2}x_n+\frac{1}{x_n}$ for 
$n = 2, 3,\ldots$

\textbf{Prove that $x_n^2>2$ for all $n>1$}

We proceed by induction. To start, consider the case where $n=2$. We see that
$x_2^2 = (\frac{1}{2}x_1+\frac{1}{x_1})^2 = (\frac{1}{2}\cdot2+\frac{1}{2})^2
= (\frac{3}{2})^2 = \frac{9}{4} > 2$.

For the inductive step we assume that $x_n^2>2$, and we want to conclude from
this that $x_{n+1}^2>2$. We proceed by manipulating the first inequality:
\[
\def\arraystretch{1.5}
\begin{array}{rl}
x_n^2>2 &\implies x_n^2-2>0 \\
&\implies (x_n^2-2)^2>0 \\
&\implies x_n^4-4x_n^2+4>0\\
&\implies \frac{1}{4}x_n^2-1+\frac{1}{x_n^2}>0\\
&\implies \frac{1}{4}x_n^2+1+\frac{1}{x_n^2}>2\\
&\implies (\frac{1}{2}x_n+\frac{1}{x_n})^2 > 2\\
&\implies x_{n+1}^2 > 2.
\end{array}
\]
Thus, by mathematical induction, $x_n^2>2$ for all $n>1$.

\textbf{Prove that $x_{n+1} < x_n$ for all $n$}

We proceed by manipulating the inequality $x_n^2>2$. However, we notice that
we have only shown this to be true for $n>1$. We will cover the case where
$n=1$ later. We will also use the fact that $x_n > 0$ for all $n$, which we can
see must be true as every $x_n$ is the sum of positive numbers in $\Q$. 
So, we can see that
\[
\def\arraystretch{1.5}
\begin{array}{rl}
x_n^2>2&\implies \frac{x_n^2}{2}>1 \\
&\implies\frac{x_n}{2}>\frac{1}{x_n} \\
&\implies x_n > \frac{1}{x_n} + \frac{x_n}{2} \\
&\implies x_n > x_{n+1}.
\end{array}
\]
We have shown that $x_{n+1}<x_n$ for $n>1$. We finish by considering $n=1$:
$x_2=\frac{3}{2}<2=x_1$. Thus, the sequence $(x_n)$ is monotone decreasing.

\textbf{Prove that $\displaystyle{\lim_{n\to\infty}x_n^2 = 2}$}

We have shown that the sequence $(x_n^2)$ is monotone decreasing
and bounded from below, specifically bounded from below by the number 2.
This sequence must therefore converge. In other words,
there exists an $L$ such that $\lim_{n\to\infty}x_n^2=L$. 
We can see also that $\lim_{n\to\infty}x_{n+1}^2=L$.
With the defintion of $x_{n+1}$ we can expand this to get 
\[\lim_{n\to\infty}\left(\frac{x_n}{2}+\frac{1}{x_n}\right)^2=
\lim_{n\to\infty}\frac{x_n^2}{4}+1+\frac{1}{x_n^2}=
\frac{L}{4}+1+\frac{1}{L}
=L.\]
Solving for $L$:
\[
\def\arraystretch{1.5}
\begin{array}{rl}
\frac{L}{4}+1+\frac{1}{L}=L &\implies 3L^2-4L-4=0 \\
&\implies (L-2)(3L+2)=0 \\
&\implies L=2 \textrm{ or } L=-\frac{2}{3}.
\end{array}
\]
Since $x_n > 0$ for all $n$, we can dismiss the negative solution. Thus,
$L=2$. In other words, $\lim_{n\to\infty}x_n^2=2$.

\subsection*{Problem 2}

Prove that if $\limsup_{n\to\infty}a_n=L$ then for all $\varepsilon>0$ there
are only finitely many $n\in\N$ for which $a_n>L+\varepsilon$.

We can prove this through contraposition. We will assume that there exists
at least one $\varepsilon>0$ such that there are infinitely many
$n$ that satisfy $a_n>L+\varepsilon$. We want to show from this that
$\limsup_{n\to\infty}a_n\neq L$. So, find such an $\varepsilon$ and fix it.
Since there are infinitely many $n$ such that $a_n>L+\varepsilon$, there exists
some subsequence $(a_{n_k})$ where for all $k$, $a_{n_k} > L+\varepsilon$.
This subsequence has been constructed such that it is bounded below by $L+\varepsilon$.

There are two possible cases for the upper bound of this subsequence. Let us first
consider the case where $(a_{n_k})$ is bounded above by some $L'\in\R$. By
the Bolzano-Weierstrass Theorem, there must be some subsequence
$(a_{n_{k_l}})$ that converges to a value in $[L+\varepsilon,L']$. This value
is greater than $L$. Since $(a_{n_{k_l}})$ is a subsequence of $a_n$, $L$ cannot
be the $\limsup$ of $a_n$ because we have shown that there exists a subsequence that
converges to a greater value.

Now consider the case where $(a_{n_k})$ is not bounded above by any real value.
This means that there must be some subsequence $(a_{n_{k_l}})$ which diverges
to $+\infty$. Since this subsequence is a subsequence of $(a_n)$, the $\limsup$
of $(a_n)$ would be $+\infty$ in this case. Thus, in both cases, 
$\limsup_{n\to\infty}a_n\neq L$, which completes our proof.

A similar argument can be used to show that
$\limsup_{n\to\infty}a_n=L$ implies that for all $\varepsilon>0$ there
exist only finitely many $n\in\N$ for which $a_n<L-\varepsilon$.

\subsection*{Section 2.1}
\subsection*{Problem 18}

Let $(a_n)$ be a sequence and let $L\in\R$. We want to prove that if every
subsequence $(a_{n_k})$ has a subsequence $(a_{n_{k_l}})$ that converges to
$L$, then the sequence $(a_n)$ converges to $L$. 

We will prove this using the contrapositive. We hope to show that if $(a_n)$ does
not converge to $L$ then there exists some subsequence $(a_{n_k})$ such that
all of its subsequences $(a_{n_{k_l}})$ fail to converge to $L$.
By the negation of the definition of convergence to $L$, we can assume that
there exists an $\varepsilon>0$ where for all $N$, there is some $n>N$ such
that $\left|a_n-L\right| \geq \varepsilon$. Find such an $\varepsilon$ and
fix it. Set $N=1$. So, there is some $n_1>1$ so that
$\left|a_{n_1}-L\right|\geq\varepsilon.$ Now consider $N=n_1$. We must also
be able to find an $n_2>n_1$ where $\left|a_{n_2}-L\right|\geq\varepsilon$.
We can continue this argument and find a sequence $(n_k)$ such that,
for all $k$, $\left|a_{n_k}-L\right|\geq\varepsilon$. This subsequence of
$(a_n)$ has been constructed such that it does not converge to $L$. More
specifically, it has been constructed such that there are no values $k$ where
$\left|a_{n_k}-L\right|<\varepsilon$. In other words, we cannot find any
subsequences of $(a_{n_k})$ that converge to $L$, because for such a
subsequence $(a_{n_{k_l}})$ to converge, there would have to be 
infinitely many values of $l$ for which
$\left|a_{n_{k_l}}-L\right|<\varepsilon$ for every $\varepsilon$.
We conclude that if $(a_n)$ does not converge to $L$, we can find a subsequence
of $(a_n)$ that does not have any subsequences that converge to $L$. So, by
contraposition, if every subsequence of $(a_n)$ has a subsequence that
converges to $L$, then $(a_n)$ converges to $L$.

\subsection*{Section 2.2}
\subsection*{Problem 25}
Make sense of the following expression as a limit and find its value:

\[ 1+\cfrac{1}{2+\cfrac{1}{2+\cfrac{1}{2+\ldots}}} \]

Define a sequence $(a_n)$ where $a_1=1+\cfrac{1}{1}=2$ and
$a_{n+1} = 1+\cfrac{1}{1+a_n}$. The value of the continued fraction above can
be interpreted as $\lim_{n\to\infty}a_n$. Assume that $(a_n)$ converges to some
value $L$, so $\lim_{n\to\infty}a_n=L$, and note that
$\lim_{n\to\infty}a_{n+1}=L$ as well. We can use the recursive definition to
expand this:
\[
\def\arraystretch{1.5}
\begin{array}{rl}
\lim_{n\to\infty}a_{n+1} &= \lim_{n\to\infty}1+\cfrac{1}{1+a_n} \\
&= 1+\cfrac{1}{1+L} = L.
\end{array}
\]

We solve for $L$:

\[
\def\arraystretch{1.5}
\begin{array}{rl}
L = 1+\cfrac{1}{1+L} &\implies L(L+1) = (L+1)+1 \\
&\implies L^2+L=L+2 \\
&\implies L^2-2=0 \\
&\implies L=\sqrt{2} \textrm{ or } L=-\sqrt{2}.
\end{array}
\]
Since each term of $(a_n)$ is positive, we can exclude the negative solution
for $L$. So we find that $L=\sqrt{2}$, which is the value of the continued fraction.
\subsection*{Problem 26}
Make sense of the following expression as a limit and find its value:

\[ \sqrt{2+\sqrt{2+\sqrt{2+\ldots}}} \]

Define a sequence $(a_n)$ where $a_1=\sqrt{2}$ and
$a_{n+1} = \sqrt{2+a_n}$. The value of the expression above can
be interpreted as $\lim_{n\to\infty}a_n$. Assume that $(a_n)$ converges to some
value $L$, so $\lim_{n\to\infty}a_n=L$, and note that
$\lim_{n\to\infty}a_{n+1}=L$ as well. We can use the recursive definition to
expand this:
\[
\def\arraystretch{1.5}
\begin{array}{rl}
\lim_{n\to\infty}a_{n+1} &= \lim_{n\to\infty}\sqrt{2+a_n} \\
&= \sqrt{2+L} = L.
\end{array}
\]

We solve for $L$:

\[
\def\arraystretch{1.5}
\begin{array}{rl}
\sqrt{2+L}=L &\implies 2+L=L^2 \\
&\implies L^2-L-2=0 \\
&\implies (L-2)(L+1)=0 \\
&\implies L=2 \textrm{ or } L=-1.
\end{array}
\]
Since each term of $(a_n)$ is positive, we can exclude the negative solution
for $L$. So we find that $L=2$, which is the value of the expression.

\subsection*{Problem 29}

Let $(a_n), (b_n)$ be sequences. Prove both of the following:

\[\limsup(a_n+b_n)\leq\limsup a_n + \limsup b_n \]
\[\liminf(a_n+b_n)\geq\liminf a_n + \liminf b_n \]

Also show that strict inequalities are possible.

To simplify explanations, define $(c_n)$ where $c_n = a_n + b_n$ for all $n$.
In addition, let $\overline{A}=\limsup a_n$, $\overline{B}=\limsup b_n$,
$\underline{A}=\liminf a_n$, and $\underline{B}=\liminf b_n$.
We can rewrite the statements as
\[\limsup c_n\leq\overline{A}+\overline{B}\]
\[\liminf c_n\geq\underline{A}+\underline{B}\]

We start by proving the first statement. We want to show that for all 
$\varepsilon>0$ there are no subsequences of $(c_n)$ that converge to a value
greater than
$\overline{A}+\overline{B}+\varepsilon$. Take any $\varepsilon_1$ such that
$0<\varepsilon_1<\varepsilon$ and let $\varepsilon_2=\varepsilon-\varepsilon_1$.
By Problem 2, there are only finitely many values of $n$ such that
$a_n>\overline{A}+\varepsilon_1$ and finitely many values of $n$ such that
$b_n>\overline{B}+\varepsilon_2$. Thus, there are only finitely many values
$n$ such that $c_n = a_n+b_n > (\overline{A}+\varepsilon_1)+
(\overline{B}+\varepsilon_2) = \overline{A}+\overline{B}+\varepsilon$. Since
the number of such $n$ values is finite, there are no subsequences of $(c_n)$
where for all $k$, $c_{n_k} > \overline{A}+\overline{B}+\varepsilon$. Thus
$\limsup c_n$ is at most $\overline{A}+\overline{B}$.

A similar argument can be used to prove the second statement by way of showing
that for all $\varepsilon > 0$ there are no subsequences of $(c_n)$ that converge
to a value less than
$\underline{A}+\underline{B}-\varepsilon$. Define $\varepsilon_1, \varepsilon_2$
the same way as above.
By Problem 2, there are only finitely many values of $n$ such that
$a_n<\underline{A}-\varepsilon_1$ and finitely many values of $n$ such that
$b_n<\underline{B}-\varepsilon_2$. Thus, the number of values $n$ where
$c_n = a_n+b_n < (\underline{A}-\varepsilon_1)+(\underline{B}-\varepsilon_2) =
\underline{A}+\underline{B}-\varepsilon$ is finite. So there are no subsequences
$(c_{n_k})$ where for all $k$, $c_{n_k} < \underline{A} + \underline{B}-\varepsilon$,
which means $\liminf c_n$ is at least $\underline{A} + \underline{B}$.

We can find an example of two sequences $(a_n), (b_n)$ for which the above statements
are true with strict inequality. Let $a_n = 1+(-1)^n$ and $b_n = -1-(-1)^n$
for all $n\in\N$. Note that $a_n=2$ for even $n$ and $a_n=0$ for odd $n$, whereas
$b_n = -2$ for even $n$ and $b_n=0$ for odd $n$. Further, we can see that
$a_n+b_n=0$ for all values of $n$. Evaluating $\liminf$ and $\limsup$ for these
sequences is straightforward

\subsection*{Problem 31}

\end{document}
